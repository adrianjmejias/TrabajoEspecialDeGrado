En el presente trabajo luego de analizar y diseñar, se concluyó satisfactoriamente la implementación de una aplicación web capaz de construir y editar visualizaciones de historiales de wikis. Gracias a la flexibilidad de Angular, se logró cumplir una arquitectura SPA que con ayuda del \textit{layout} de Angular Material se implementaron componentes visuales bastante adaptativos a distintas densidades de pantallas. Se diseñaron y ofrecieron visualizaciones generales para cada artículo extraído haciendo uso del API de Wikimetrics, entre varias la más relevante es la gráfica de Wiki History Flow. Es importante persistir el progreso y acciones de los usuarios, por lo tanto se diseñó e implementó un API de Usuarios capaz de proveer los recursos necesarios, apoyándose de una base de datos no relacional.

Angular a su vez dispone de una herramienta llamada Angular CLI que ofrece diversas funcionalidades para facilitar la construcción, reproducción y despliegue de la aplicación para distintos ambientes de trabajos (desarrollo o producción). Por lo tanto, se pudo servir en un ambiente de desarrollo exitosamente.

Se aplicó correctamente una metodología ágil, que fue llevada totalmente en la plataforma Github, en donde la pizarra (como artefacto de la metodología Kanban) jugó un papel importante, la cual se encargaba de proyectar el estado de las asignaciones o tareas. Además, los \textit{issues} prestaban un espacio para abrir discusión y documentar toma de decisiones.

\section{Limitaciones}

\begin{itemize}
    \item Principalmente se iba a usar el API de Wikipedia para realizar la autenticación de usuarios y extraer los artículos de su \textit{watchlist} como primera instancia, pero Wikipedia dejó obsoleta la manera vieja de autenticación y la nueva forma es mediante OAuth2, por lo que pedían usa serie de requerimientos que estaban fuera del alcance. Por tal motivo se optó por gestionar los usuarios con nuestro propio sistema.
    
    \item Para la realización de visualizaciones se tuvo que diseñar e implementar un nuevo end-point en el API de Wikimetrics debido a que los otros eran pocos flexibles. Por lo tanto se tomó tiempo del trabajo para el análisis de la misma. Con el nuevo end-point por falta de tiempo los queries en MongoDB los tuvo que definir el front-end en base a los requerimientos.
\end{itemize}

\section{Trabajos futuros}

\begin{itemize}
    \item Dejar de manejar los usuarios internamente y manejar la autenticación con el API de Wikipedia. Lo principal sería cumplir con los requerimientos que pide MediaWiki como API para crear la aplicación usando OAuth2. De esta forma solo manejaremos un inicio de sesión usando el usuario de Wikipedia.
    
    \item Habilitar una opción para poder sincronizar los artículos del watchlist del usuario de Wikipedia con los artículos de esta plataforma.
    
    \item Analizar qué visualizaciones pueden ser frecuentes en los usuarios y ofrecerlas como visualizaciones predeterminadas.
    
    \item Implementar la gráfica original de History Flow. Esta visualización requiere de mucho cómputo, por lo que calcularlas en el front-end es inviable. La idea es delegarle esta funcionalidad a un servicio en el back-end que envíe los datos preparados para la visualización.
    
    \item Alguna funcionalidad para compartir \textit{dashboard} de visualizaciones o visualizaciones individuales en modo de solo lectura
    
    \item Implementar un servicio en el backend que se encargue de actualizar el estado de los artículos extraídos en el API de Usuarios. De esta manera, se evitan constantes peticiones HTTP en la aplicación web (HTTP Polling).
\end{itemize}

\section{Contribuciones}
\begin{itemize}
    \item Se realizó una investigación de bibliotecas de visualización que puede servir como apoyo para la decisión de futuros trabajos dependiendo de los requerimientos.
    
    \item Se realizó investigación en el área de Visualización de Datos, refrescando conceptos teóricos claves y planteamientos de como resolver problemas haciendo uso de diferentes técnicas.
    
    \item Se implementó un componente en Angular que facilita la creación de visualizaciones responsives usando Plotly, en donde será de apoyo para aquellos trabajos que involucren las mismas tecnologías.
    
    \item Se elaboró un componente de Angular que construye la gráfica Wiki History Flow haciendo uso de D3. Para trabajos futuros puede servir de apoyo para la mejora o construcción de la misma.
    
    \item Al ser una aplicación web que implica tecnologías nuevas e interesantes puede aportar contenido y ejemplos para la materia de Aplicaciones en Internet.
\end{itemize}