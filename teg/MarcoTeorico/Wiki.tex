\section{Wiki}

El término wiki proviene de la raiz hawaiana wiki, que significa "rápido", y fue propuesto por Ward Cunningham, quien a su vez define los sitios web wiki como "La base de datos más simple que puede existir" [Cunningham, Ward (June 27, 2002), What is a Wiki]. Con el tiempo el concepto de Wiki fue evolucionando, y hoy en dia cuando hablamos de wiki nos referimos a un sitio web que permite a sus usuarios colaborar en su estructura y contenido.\\

Wikimedia es el nombre colectivo del movimiento wikimedia, que incluye un grupo de proyectos interrelacionados, tales como: Wikipedia, Wiktionary, Wikiquote, Wikibooks, Wikisource, entre otros, cuyo proposito es usar el poder colaborativo de internet, y el concepto wiki, para compartir conocimiento gratuito de cualquier tipo.\\

MediaWiki es el motor que impulsa los sitios web basados en wiki. En este documento se hará efasis en este sistema, debido a que se trabajará con articulos de Wikipedia, quien hace uso de Mediawiki para cumplir con muchas de sus funcionalidades.






