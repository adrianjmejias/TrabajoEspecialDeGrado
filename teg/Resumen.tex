\section*{Resumen}

% Tıtulo:
% Desarrollo de un editor de visualizaciones de propiedades de historiales de
% wikis.
% Autor:
% Leonardo Testa.
% Tutor:
% Prof. Eugenio Scalise.
% Un Wiki es un sitio web, generalmente de car ́acter informativo (como lo es
% Wikipedia), que puede ser modificado por m ́ultiples personas. Cada una de
% estas modificaciones son almacenadas, y en conjunto conforman un historial
% de versiones, en donde cada versi ́on representa una modificaci ́on y los efectos
% que caus ́o en el art ́ıculo wiki. Siendo Wikipedia un caso real con bastante

% popularidad, es normal que el historial de versiones de un art ́ıculo sea su-
% ficientemente extenso y complejo, por lo tanto las personas interesadas en

% mantener el art ́ıculo “sano” perder ́an una gran suma de tiempo revisando
% las modificaciones. En este documento, presentaremos la investigaci ́on y la
% realizaci ́on de una herramienta web que facilita la lectura de propiedades del
% historial a aquellas personas interesadas, en donde se optar ́a por visualizaci ́on

% de datos como estrategia, de esta forma, mediante una interfaz capaz de ma-
% nipular gr ́aficas el usuario podr ́a proyectar distintas propiedades y conseguir

% f ́acilmente informaci ́on m ́as completa y concretar patrones.
% Palabras claves:
% Visualizaci ́on de datos, wiki, propiedades de historiales, gr ́aficas, herramienta
% web, editor de visualizaciones, wikipedia.



