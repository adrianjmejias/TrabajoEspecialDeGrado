
\section{Contexto}

La wikipedia puede aportar mucha mas data que la aportada por


\section{Objetivo general}
Consumir y extender la API de wikimetrics para desarrollar una aplicacion web que habilite a sus usuarios construir y visualizar graficas

\section{Objetivos específicos}
\begin{itemize}{}{}
    \item Definir los requerimientos de la aplicacion
    \item Implementar un
    \item Implementar una applicacion web responsive que ofrezca las funcionalidades requeridas por un watcher de un wiki y que pueda ser reconocida por los motores de busqueda.
    \item Utilizar un metodo ´agil para el desarrollo de la aplicaci´on.
    \item Realizar el despliegue y puesta en producci´on de la aplicaci´on
\end{itemize}


\section{Justificación}
Wikipedia contiene en si una masiva cantidad de datos "base" - como articulos, eventos, noticias, media y demas. Pero tambien  tambien ella misma genera nuevos datos con cada nueva adicion y edicion a su haber documental. Todos estos "rastros" que dejan miles de colaboradores dia a dia son conocidos como metadata.

Esta desde luego tiene un inmenso valor por si misma, y es refinada principalmente por analistas de datos y aficionados que quieren buscar patrones, relaciones o informacion que no es facil o posible de distinguir con ojos humanos.

Nuestra labor entonces es facilitar este trabajo para ellos dejandoles crear sus propias graficas sobre los articulos que deseen


\section{Distribución del documento}

*** Pongamos esto al terminar ***

