\section{Marco Tecnológico}



\subsection{Librerias o frameworks para aplicaciones intensivas de frontend }

\section{Arquitectura}

Para que una aplicacion sea descubierta y usada por internautas es fundamental que tenga una buena relacion con los motores de busqueda.

Sin embargo tambien para asegurar la larga vida y mantenibilidad de la aplicacion y la facilidad de desarrollo se debe tomar en cuenta herramientas extensamente empleadas contemporaneamente como Angular, React y Vue.

El problema entonces recae en que estas tecnologías son meramente para SPA. Lo que implica entonces que no existe una nocion "real" de seo - En las SPA el enrutamiento ocurre del lado del cliente usando javascript, y en consecuencia los crawlers de los motores de busqueda no saben interpretar estas paginas.

Como remedio surge un nuevo paradigma, que es el que vamos a usar para esta aplicacion, conocido como Server Side Rendering; donde se utiliza estas tecnologias SPA como un motor de plantillas para retornar un HTML que los motores de busqueda puedan entender, y despues por un proceso conocido como hydration, las aplicaciones en el lado del cliente dejan de comportarse como HTML plano y retoman sus funcionalidades de SPA.

Asi entonces llegamos al perfecto balance en el que tenemos herramientas actuales y faciles de usar, que tambien cumplen con los requerimientos de los motores de busqueda para indexar nuestras paginas.

\section{Librerias para la visualización de datos}

\subsection{ Data Driven Documents (D3) }

\subsection{ Recharts }

\subsection{  }

