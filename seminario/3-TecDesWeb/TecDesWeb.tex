El propósito de todo sitio web es ser utilizado y aprovechado por la mayor cantidad de personas posibles, y es por eso que el punto clave del desarrollo web se encuentra en cómo los desarrolladores proporcionan un sitio web que cumpla con ese propósito. Para conseguir dicho propósito se deben tener múltiples consideraciones que abarca desde la experiencia de usuario hasta la calidad de la infraestructura y el código realizado. Cuando hablamos de experiencia de usuario nos referimos al resultado de la interacción del usuario con los distintos elementos que posee el sitio web, de tal forma que el usuario pueda tener una experiencia positiva al hacer uso de dichos elementos. Petter Morville \cite{UXFactors} propone que para lograr una buena experiencia de usuario se debe tener en cuenta el correcto uso de los siguientes factores:

\begin{itemize}
  \item Utilidad: el producto realizado debe ser de utilidad para el público objetivo.
  \item Usabilidad: debe ser un sistema que sea simple y fácil de usar para el usuario.
  \item Deseabilidad: cuando un producto es deseable, los usuarios se sienten atraídos a el.
  \item Encontrable: se debe construir un sistema que sea navegable y fácil de encontrar, de tal forma que los usuarios puedan encontrar lo que necesiten fácilmente.
  \item Accesibilidad: el sistema debe poder ser usado por la mayor cantidad de personas posibles.
  \item Credibilidad: el sistema debe proporcionar seguridad al usuario acerca de la credibilidad de su contenido
  \item Valor: una experiencia de usuario valiosa implica un buen uso de los demás factores.
\end{itemize}

Sin embargo, para que sea posible cumplir con todos los factores que permiten una buena experiencia de usuario,
es indispensable realizar un código de calidad que proporcione al sitio web un buen rendimiento, permitiendo al sitio web
desempeñar su función correctamente.

En este capítulo presentaremos las herramientas que nos facilitaran el trabajo para lidiar con las problemáticas mencionadas.

\section{Arquitectura}

Para que una aplicación sea descubierta y usada por internautas es fundamental que tenga una buena relación con los motores de búsqueda.

Sin embargo también para asegurar la larga vida y mantenibilidad de la aplicación y la facilidad de desarrollo se debe tomar en cuenta herramientas extensamente empleadas contemporáneamente como Angular, React y Vue.

El problema entonces recae en que estas tecnologías son meramente para SPA. Lo que implica entonces que no existe una noción "real" de seo - En las SPA el enrutamiento ocurre del lado del cliente usando javascript, y en consecuencia los crawlers de los motores de búsqueda no saben interpretar estas paginas.

Como remedio surge un nuevo paradigma, que es el que vamos a usar para esta aplicación, conocido como Server Side Rendering; donde se utiliza estas tecnologías SPA como un motor de plantillas para retornar un HTML que los motores de búsqueda puedan entender, y después por un proceso conocido como hydration, las aplicaciones en el lado del cliente dejan de comportarse como HTML plano y retoman sus funcionalidades de SPA.

Asi entonces llegamos al perfecto balance en el que tenemos herramientas actuales y fáciles de usar, que también cumplen con los requerimientos de los motores de búsqueda para indexar nuestras paginas.


\section{Tecnologías para el desarrollo}

Actualmente existe un catalogo muy amplio de tecnologías de desarrollo web, por lo que escoger las debidas herramientas y sobre todo analizar las compatibilidades entre estas herramientas pueden ser el punto clave para obtener la mejor eficiencia y experiencia de usuario.

Para nuestro proyecto haremos uso de un servidor cuyo propósito será obtener los datos y transformarlos de forma que el cliente los pueda usar directamente sin necesidad usar mucho computo, esto con el propósito de proporcionar una mejor experiencia de usuario.

El este capitulo hablaremos de las herramientas de desarrollo web usadas para llevar a cabo el proyecto, y las descompondremos en 2 categorías: Servidor y Cliente.

\section{Servidor}

\subsection{Fastify}

Fastify es un framework web para Node.js de código abierto concentrado en proporcionar el mejor rendimiento, y una arquitectura flexible. \\

Las principales caracteristicas con las que fastify fue creado son:

\begin{itemize}

    \item Alto rendimiento: Si comparamos la velocidad de Fastify con otros frameworks web como Express, tal como podemos ver en la figura \ref{fig:fastify_vs_express}, notamos que Fastify es aproximadamente un 20\% más rápido que Express.
    \item Extensible: Fastify es completablemente extensible mediante el uso de hooks, puglins o decoradores
    \item Basado en esquemas: incluso si no es obligatorio, recomendamos usar JSON Schema para validar sus rutas y serializar sus salidas, internamente Fastify compila el esquema en una función de alto rendimiento.
    \item Registro: 
    \item Amigable para los desarrolladores:
    \item Listo para usar con Typescript:
  
\end{itemize}


\begin{figure}[H]
    \includegraphics[scale=0.5]{fastify_vs_express.png}
    \caption{ Número de peticiones por segundo para distinta cantidad de conexiones}
    \label{fig:fastify_vs_express}
\end{figure}

\subsection{MongoDB}

MongoDB es una sistema de base de datos de documentos NoSQL escalable y flexible diseñado para lidiar con los conflictos que poseen las bases de datos relacionales y las limitaciones de otras soluciones NoSQL. En lugar de guardar los datos en tablas, tal y como se hace en las bases de datos relacionales, MongoDB guarda estructuras de datos BSON (una especificación similar a JSON) con un esquema dinámico, lo que facilita la integración de los datos.

Uno de los principales beneficios del uso de base de datos NoSQL antes que SQL es la facilidad de manejo de los datos, ya que solo se consultan documentos sin necesidad de realizar operaciones SQL complejas que requieren conocimiento previo de la estructura de la base de datos. Ademas, debido a la similitud entre la especificación BSON y JSON, las consultas a la base de datos resultan transparentes en comparación a su contraparte SQL, el cual necesita de un algoritmo para convertir los datos tabulares en algo que pueda ser comprendido por el lenguaje de programación usado en el servidor backend.

Como se puede observar en la imagen \ref{fig:db-ranking}, mongoDB es actualmente la opción mas usada entre la categoria de bases de datos NoSQL, ocupando el puesto numero 5 en el ranking de bases de datos. Esto nos proporciona suficiente seguridad de que existirá soporte para las aplicaciones que hagan uso de MongoDB por los proximos años.

\begin{figure}[H]
  \includegraphics[scale=.355]{db-ranking.jpg}
  \caption{Ranking de bases de datos mas populares de 2022}
  \label{fig:db-ranking}
\end{figure}

Algunas de las características más importantes de MongoDB son:

\begin{itemize}

  \item Consultas ad-hoc optimizadas para análisis en tiempo real
  \item Indexación adecuada para mejores ejecuciones de consultas
  \item Replicación para una mejor disponibilidad y estabilidad de los datos
  \item Fragmentación
  \item Balanceo de carga

\end{itemize}

Para realizar operaciones en la base de datos MongoDB es necesario de instalar un MongoDB Driver compatible con NodeJS, el cual permite establecer una conexión con la base de datos, y provee una API que facilita la interacción con la misma. A continuación se mostrará un código básico de conexión y consulta de datos usando el NodeJs MongoDB Driver.

\begin{lstlisting}
    import { MongoClient } from "mongodb";

    const uri = "<connection string uri>";

    const client = new MongoClient(uri);

    async function run() {

        await client.connect();

        const database = client.db("sample_mflix");

        const movies = database.collection("movies");

        const query = { title: "The Room" };

        const options = {};

        const movie = await movies.findOne(query, options);

    }
\end{lstlisting}

En el código se puede observar como se usa el cliente de MongoDB para conectarse a la base de datos \textquote{sample\_mflix} y después consultar todas las películas cuyo título sea igual a \textquote{The Room}.

\section{Cliente}

Para la realización de este proyecto se usarán las siguientes tecnologías:

\subsection{ReactJS}

ReactJS es una librería de JavaScript de código abierto desarrollada por Facebook para facilitar la creación de componentes interactivos, reutilizables, para desarrollos de interfaces de usuario, especialmente aplicaciones de una sola página.\\

React maneja el concepto de \say{programación reactiva} haciendo uso de un DOM Virtual, lo le permite determinar qué partes del DOM han cambiado comparando contenidos entre la versión nueva y la almacenada den el DOM virtual, para así propagar los datos generando cambios en la aplicación, es decir, los datos \say {reaccionan} ejecutando una serie de eventos.\\

Este concepto de reactividad es lo que hace a la librería altamente eficiente, ya que limita la actualización del DOM solamente a los elementos que han cambiado.\\

Otras características que destacan en React son:\\

\begin{itemize}
  \item \textbf{Componentes} \hfill \\

        El código de React es hecho con entidades llamadas componentes. Los componentes pueden ser renderizados en elementos particulares del DOM usando la librería de React DOM. Estos componentes son capaces de recibir parámetros conocidos como "propiedades del componente" de la siguiente forma: \hfill \\

        \begin{lstlisting}
                ReactDOM.render(<Greeter greeting="Hello World!" />, document.getElementById('myReactApp'));
            \end{lstlisting}

        Las 2 formas de declarar componentes en react es mediante el uso de funciones o clases, y generalmente se usa una de las dos opciones de forma situacional.

  \item \textbf{JSX} \hfill \\

        JSX, también llamado Javascript XML, es una extension a la sintaxis del lenguaje javascript. Este provee una forma de estructurar componentes usando una sintaxis familiar para muchos desarrolladores. Los componentes de React son usualmente escritos usando JSX, aunque también pueden ser escritos usando Javascript puro.

        Un ejemplo de código JSX:

        \begin{lstlisting}
                class App extends React.Component {
                    render() {
                        return (
                        <div>
                            <p>Header</p>
                            <p>Content</p>
                            <p>Footer</p>
                        </div>
                        );
                    }
                }
            \end{lstlisting}

  \item \textbf{Hooks} \hfill \\

        Los hooks son funciones que permiten a los desarrolladores \say{engancharse} a los estados de React y a ciertos puntos dentro del ciclo de vida de los componentes.

        React proporciona algunos hooks integrados tales como: useState, useContext, useReducer, useMemo y useEffect, los cuales son los mas usados y permiten controlar los estados y eventos respectivamente.


\end{itemize}

\subsection{Next.js}

Next.js es un framework desarrollado encima de Node.js que permite a las aplicaciones de React usar funcionalidades como el renderizado del lado servidor y la generación de paginas web estáticas.\\

Por defecto, Next.js pre-renderiza cada pagina. Esto significa que Next.js genera HTML para cada pagina en adelanto, en vez de hacerse con Javascript del lado del cliente. Pre-renderizado puede resultar en mejor rendimiento y SEO.\\

Cada HTML generado es asociado con el mínimo código Javascript necesario para que funcione la pagina. Cuando una página es cargada en el explorador, su código javascript se ejecuta y hace la página totalmente interactiva. A este proceso de le conoce como \say{hydration}

Next.js ofrece 2 formas de pre-renderizado:

\begin{itemize}
  \item Generación estática: El HTML es generado a tiempo de ejecución y será reutilizado en cada petición.
  \item Renderizado lado servidor: El HTML es generado en cada petición
\end{itemize}

\subsection{Material UI}

Material UI es un framework de React creado con el objetivo de proporcionar una forma sencilla a los desarrolladores de aplicar los principios de Material Design \cite{MaterialDesignPrinciples} usando componentes de React. Este framework se destaca por la gran variedad de componentes disponibles, que van desde elementos visuales sencillos como botones o selectores, hasta complejos como modales o drawers. Además de su variedad de componentes, Material UI tambien destaca por su gran cantidad de contribuyentes, lo que asegura la longevidad y mantenibilidad del framework.

Entre las principales caracteristicas que posee Material UI se encuentran:

\begin{itemize}
  \item Posibilidad de crear temas, lo que permite modificar los estilos de los componentes predeterminados de la librería de forma global en caso de necesitarse un diseño más personalizado.
  \item Construido usando la estrategia móvil primero, en la cual se crea el código primero para los dispotivos móviles, y despues se escalan usando las reglas de CSS necesarias.
  \item Los componentes de Material UI se consideran autosuficientes, por lo que solo utilizan los estilos CSS necesarios para mostrar el componente en pantalla, es decir, no dependen de hojas de estilos globales como en el caso de normalize.css o bootstrap
\end{itemize}

A continuación se monstrará el uso de algunos componentes que ofrece la libreria:

\begin{itemize}
  \item Botones\\

        \begin{lstlisting}
                <Button variant="text">Text</Button>
                <Button variant="contained">Contained</Button>
                <Button variant="outlined">Outlined</Button>
                <Button color="secondary">Secondary</Button>
                <Button size="small">Small</Button>
                <IconButton aria-label="delete">
                    <DeleteIcon />
                </IconButton>
            \end{lstlisting}

        Material UI viene incluido con 3 variantes visuales de botones, los cuales son usados dependiendo de la importancia o enfasis que se le quiera dar al botón. Además, tambien posee variaciones de color y tamaño predeterminados y la posibilidad de usar íconos que funcionan como botones.

  \item Selectores\\

        \begin{lstlisting}
                <FormControl fullWidth>
                    <InputLabel id="demo-simple-select-label">Age</InputLabel>
                    <Select
                        labelId="demo-simple-select-label"
                        id="demo-simple-select"
                        value={age}
                        label="Age"
                        onChange={handleChange}
                    >
                        <MenuItem value={10}>Ten</MenuItem>
                        <MenuItem value={20}>Twenty</MenuItem>
                        <MenuItem value={30}>Thirty</MenuItem>
                    </Select>
                </FormControl>
            \end{lstlisting}

        El selector comunmente se encuentra envuelto en un compontente de FormControl, el cual provee contexto para hacer uso de campos requeridos, manejo de errores y uso de propiedades como seleccionado o campo lleno.

  \item Sliders\\

        \begin{lstlisting}
                const marks = [
                    {
                        value: 0,
                        label: "0km"
                    },
                    {
                        value: 20,
                        label: "20km"
                    },
                    {
                        value: 37,
                        label: "37km"
                    },
                    {
                        value: 100,
                        label: "100km"
                    },
                ];
                <Slider aria-label="Volume" value={25} onChange={handleChange} />
                <Slider defaultValue={30} step={10} marks min={10} max={110} disabled />
                <Slider
                    aria-label="Custom marks"
                    defaultValue={20}
                    getAriaValueText={valuetext}
                    step={10}
                    valueLabelDisplay="auto"
                    marks={marks}
                />
            \end{lstlisting}

        Material UI permite customizar los sliders de mútiples formas, tales como usar valores discretos, cambiar el tamaño del slider, o utilizar marcas personalizadas para el slider, lo que permite dar contexto al usuario sobre los valores mostrados en el slider.

\end{itemize}

