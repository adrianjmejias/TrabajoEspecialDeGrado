\subsection{Contexto}

% Resumen del estado actual de wikipedia. Situación alrededor la tesis tiene sentido
% Me va a situar en el problema que sirve de insumo para justificar el TEG

% DEBE identificar el problema, resuelto por el objetivo general

Con aproximadamente seis millones cuatrocientos mil artículos ( 6.400.000 ) la Wikipedia lidera como la enciclopedia mas extensa del mundo. No solo eso, sino que muy comúnmente al googlear sobre algún tema de interés siempre es el primer resultado.
Esto es gracias al esfuerzo colaborativo de miles de 




% Se hablan de soluciones previas y en eso introducimos wikimetrics y el front

Y son estos mismos grupos que han desarrollado soluciones para extender y analizar las wikis.




% presentamos nuestra solución

Nuestra labor entonces es facilitar este trabajo para ellos dejándoles crear sus propias gráficas sobre los artículos que deseen

\section{Justificación}

Wikipedia contiene en si una masiva cantidad de datos "base" - como artículos, eventos, noticias, media y demás. Pero también ella misma genera nuevos datos con cada nueva adición y edición a su haber documental. Todos estos "rastros" que dejan miles de colaboradores dia a dia son conocidos como metadata.

Esta desde luego tiene un inmenso valor por si misma, y es refinada principalmente por analistas de datos y aficionados que quieren buscar patrones, relaciones o información que no es fácil o posible de distinguir solamente viendo números y fechas.

% TODO: Porque este trabajo es un trabajo util
% Aprovechar la data que genera wikipedia
% wikipedia tiene apis
% sacarle provecho
% apoyar a los watchers en su labor de vigilancia de sus artículos


\section{Metodologías ágiles}

    \section{Frameworks}

    \begin{enumerate}
        \item Kanban: Tiene como objetivo la mejora continua, la flexibilidad en la gestión de tareas y un flujo de trabajo mejorado. Con este enfoque ilustrativo, el progreso de todo el proyecto se puede comprender fácilmente de un vistazo. Para esto hace uso del tablero Kanban, que es una herramienta que visualiza todo el proyecto para rastrear el flujo de su proyecto. A través de este enfoque gráfico de los tableros Kanban, un miembro nuevo o una entidad externa puede comprender lo que está sucediendo en este momento, las tareas completadas y las tareas futuras.
        \item Scrum: Es un framework para desarrollo, entrega, y mantenimiento de proyectos en un ambiente complejo, con un enfasis inicial en el desarrollo de software, aunque también ha sido utilizado en otras areas como la investigación, ventas, mercadeo y tecnologías avanzadas. Esta diseñado para equipos de 10 personas o menos, quienes rompen su trabajo en metas que pueden ser completadas en iteraciones de tiempo fijo, llamadas \emph{sprints}, con duraciones aproximadas de 2 semanas. 
        \item Lean software development: Es un framework popular basado en optimizar tiempo de desarrollo y recursos, eliminando desperdicios y entregando solamente lo que el producto necesita. El método Lean es usualmente referido como la estrategia del \say{Producto Minimo Viable (PMV)}, 
        estrategia, en la que un equipo lanza una versión mínima de su producto al mercado, aprende de los usuarios lo que les gusta, lo que no les gusta y lo que quieren que se agregue, y luego itera en función de estos comentarios.
        \item Extreme programming (XP): Es una metodología de desarrollo de software cuyo objetivo es mejorar la calidad del software y la adaptabilidad al cambio de los requerimientos del cliente. Al ser un tipo de metodología ágil, se basa en el uso de ciclos de desarrollo cortos con lanzamientos frecuentes, con el propósito de de mejorar la productividad e introducir \say{checkpoints} en los que se puedan adoptar nuevos requisitos de clientes.
        \item Adaptive Software Development (ASD): Es una consecuencia directa del desarrollo agil. Su objetivo es permitir que los equipos se adapten rápida y eficazmente a los requisitos cambiantes o las necesidades del mercado mediante la evolución de sus productos con una planificación ligera y un aprendizaje continuo. El enfoque ASD alienta a los equipos a desarrollarse de acuerdo con un proceso de tres fases: especular, colaborar, aprender. 
        \item Rapid application development (RAD):  es una forma de metodología de desarrollo de software ágil que prioriza las versiones e iteraciones rápidas de prototipos. A diferencia del método Waterfall, RAD enfatiza el uso de software y los comentarios de los usuarios sobre la planificación estricta y el registro de requisitos.
    \end{enumerate}



RANDOM TEXT FOUND IN PROPUESTA
    Sin embargo existe una oportunidad para aprovechar mejor la metadata de wikipedia, y explorar y explotar esta sera el problema a resolver de este trabajo especial de grado