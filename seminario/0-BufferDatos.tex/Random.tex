\subsection{Contexto}

% Resumen del estado actual de wikipedia. Situación alrededor la tesis tiene sentido
% Me va a situar en el problema que sirve de insumo para justificar el TEG

% DEBE identificar el problema, resuelto por el objetivo general

Con aproximadamente seis millones cuatrocientos mil artículos ( 6.400.000 ) la Wikipedia lidera como la enciclopedia mas extensa del mundo. No solo eso, sino que muy comúnmente al googlear sobre algún tema de interés siempre es el primer resultado.
Esto es gracias al esfuerzo colaborativo de miles de 




% Se hablan de soluciones previas y en eso introducimos wikimetrics y el front

Y son estos mismos grupos que han desarrollado soluciones para extender y analizar las wikis.




% presentamos nuestra solución

Nuestra labor entonces es facilitar este trabajo para ellos dejándoles crear sus propias gráficas sobre los artículos que deseen

\section{Justificación}

Wikipedia contiene en si una masiva cantidad de datos "base" - como artículos, eventos, noticias, media y demás. Pero también ella misma genera nuevos datos con cada nueva adición y edición a su haber documental. Todos estos "rastros" que dejan miles de colaboradores dia a dia son conocidos como metadata.

Esta desde luego tiene un inmenso valor por si misma, y es refinada principalmente por analistas de datos y aficionados que quieren buscar patrones, relaciones o información que no es fácil o posible de distinguir solamente viendo números y fechas.

% TODO: Porque este trabajo es un trabajo util
% Aprovechar la data que genera wikipedia
% wikipedia tiene apis
% sacarle provecho
% apoyar a los watchers en su labor de vigilancia de sus artículos

