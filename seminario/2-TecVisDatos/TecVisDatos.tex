Para el análisis de datos, el paso fundamental es la visualización de estos, dado que sin una imagen clara de la información a trabajar, los especialistas no pueden comparar grupos de datos de forma eficiente. 

"La visualización de datos es la práctica de traducir información en un contexto visual, como un mapa o gráfico, para facilitar que el cerebro humano comprenda y extraiga información útil". \cite{DefinitionDataViz}

De esta manera se facilita la identificación, localización de errores, patrones y cualquier punto a resaltar para su próxima conclusión.

Esta tarea además de importante es compleja por lo cual necesita de agrupaciones que se encarguen de trabajar en ella constantemente. Estas agrupaciones publican sus esfuerzos en plataformas como Github para así poder aportar a la investigación, discutir soluciones, revisar problemas y atraer nuevos colaboradores que permitan el crecimiento y consiguiente eficacia del trabajo.

En esta sección se busca investigar y comparar esos productos probados, refinados y simplificados para dar con la que mejor se adapte para los requerimientos a cubrir. 

\section{Librerías o frameworks para aplicaciones intensivas de frontend}

%LATE_TODO

\section{Librerías para la visualización de datos}
Pero no cualquier librería 
Para la selección de librería se consideran los siguientes factores:

\begin{enumerate}
    \item {Debe ser un proyecto open source.}
    \item {De ser posible, debe tener bindings para ReactJS para facilidad en el desarrollo.}
    \item {Debe ser extensible para poder implementar aquellas visualizaciones que sean muy específicas.}
    \item {Deben ser longevas y tener cierta garantía de que será mantenida en el tiempo,
    asegurando así que para futuras mejoras de este TEG sigan disponibles.}
    \item {Debe ser fácil de utilizar}
\end{enumerate}

\subsection{ Data Driven Documents (D3) }
\begin{itemize}
    \item URL de repositorio \href{https://github.com/d3/d3}{https://github.com/d3/d3}
    \item Mantenida por la organización homónima, d3
\end{itemize}

\subsection{ echarts }
\begin{itemize}
    \item URL de repositorio \href{https://github.com/apache/echarts}{https://github.com/apache/echarts}
    \item Mantenida por Apache
\end{itemize}

\subsection{ Comparación de bibliotecas }

Después de comparar estas librerías, sus APIs y leer otras investigaciones la decisión final es utilizar echarts

La base de esta conclusión es la simplicidad \cite{EchartsDecision}, mientras que d3 permite una gran flexibilidad, la curva de aprendizaje tiene una inclinación pronunciada. 
En comparación, echarts se ve como una herramienta plug and play que requiere mínima configuración y ahorra al desarrollador días de esfuerzo.

El tipo de gráficas a utilizar son las siguientes
\begin{itemize}    
    \item Número
    \item Línea
    \item Barra
    \item Área
    \item Torta
    \item Dispersión
    \item History flow
\end{itemize}

Para estas gráficas echarts provee soluciones pre-hechas y si es el caso de gráficos de interés en nichos específicos, como es el caso del History Flow, permite que fácilmente crees gráficas completamente nuevas. 
También provee gráficos de Sankey y Stacked Area Chart que son bastante parecidos y podrían ser personalizados para parecerse al History Flow. 

\section{ Proyectos alternos }
Estos dos gigantes en el mundo de la visualización de datos por si solos son opciones excelentes, pero trabajan sobre JavaScript vainilla y no están adaptadas para trabajar directamente sobre tecnologías como ReactJS. 
Para facilidad del desarrollo de la aplicación se utilizará una librería que se encargue de adaptar echarts a ReactJS.

\subsection{ echarts for react }
\begin{itemize}
    \item URL del repositorio \href{https://github.com/hustcc/echarts-for-react}{https://github.com/hustcc/echarts-for-react}
\end{itemize}