\section{Marco Tecnológico}

\subsection{Librerías o frameworks para aplicaciones intensivas de frontend }

\section{Arquitectura}

Para que una aplicación sea descubierta y usada por internautas es fundamental que tenga una buena relación con los motores de búsqueda.

Sin embargo también para asegurar la larga vida y mantenibilidad de la aplicación y la facilidad de desarrollo se debe tomar en cuenta herramientas extensamente empleadas contemporáneamente como Angular, React y Vue.

El problema entonces recae en que estas tecnologías son meramente para SPA. Lo que implica entonces que no existe una noción "real" de seo - En las SPA el enrutamiento ocurre del lado del cliente usando javascript, y en consecuencia los crawlers de los motores de búsqueda no saben interpretar estas paginas.

Como remedio surge un nuevo paradigma, que es el que vamos a usar para esta aplicación, conocido como Server Side Rendering; donde se utiliza estas tecnologías SPA como un motor de plantillas para retornar un HTML que los motores de búsqueda puedan entender, y después por un proceso conocido como hydration, las aplicaciones en el lado del cliente dejan de comportarse como HTML plano y retoman sus funcionalidades de SPA.

Asi entonces llegamos al perfecto balance en el que tenemos herramientas actuales y fáciles de usar, que también cumplen con los requerimientos de los motores de búsqueda para indexar nuestras paginas.

\section{Librerías para la visualización de datos}

Para la selección de librería se consideran los siguiente factores

1. Debe ser un proyecto open source.
2. De ser posible, debe tener bindings para react para facilidad en el desarrollo.
3. Debe ser extensible para poder implementar aquellas visualizaciones que sean muy específicas.
4. Deben ser longevas y tener cierta garantía de que sera mantenida en el tiempo, asi se asegura que 

\subsection{ Data Driven Documents (D3) }




\subsection{ Recharts }





\subsection{  }

