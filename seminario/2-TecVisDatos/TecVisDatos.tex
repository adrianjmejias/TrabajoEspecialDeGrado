\section{Librerías o frameworks para aplicaciones intensivas de frontend}

\section{Librerías para la visualización de datos}
Para la selección de librería se consideran los siguiente factores

\begin{enumerate}
    \item {Debe ser un proyecto open source.}
    \item {De ser posible, debe tener bindings para ReactJS para facilidad en el desarrollo.}
    \item {Debe ser extensible para poder implementar aquellas visualizaciones que sean muy específicas.}
    \item {Deben ser longevas y tener cierta garantía de que sera mantenida en el tiempo, asegurando así.}
\end{enumerate}

\subsection{ Data Driven Documents (D3) }
\begin{itemize}
    \item URL de repositorio \href{https://github.com/d3/d3}{https://github.com/d3/d3}
    \item Fecha de ultima versión: 11 de Abril de 2022
    \item Mantenida por la organización homónima
\end{itemize}

\subsection{ echarts }
\begin{itemize}
    \item URL de repositorio \href{https://github.com/apache/echarts}{https://github.com/apache/echarts}
    \item Fecha de ultima versión
    \item Mantenida por Apache
\end{itemize}

\section{ Proyectos alternos }
Estos dos gigantes en el mundo de la visualización de datos por si solos son opciones excelentes,
pero son opciones que trabajan sobre javascript vainilla y no están adaptadas para trabajar directamente sobre tecnologías como ReactJS.
Para facilidad del desarrollo se utilizará una librería que se encargue de hacer esta adaptación por nosotros.

\subsection{ recharts }
\begin{itemize}
    \item URL de repositorio \href{https://github.com/recharts/recharts}{https://github.com/recharts/recharts}
    \item Fecha de última versión: 1 de Abril de 2022
\end{itemize}