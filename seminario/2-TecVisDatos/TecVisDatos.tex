Para el análisis de datos, el paso fundamental es la visualización de datos, sin este paso los analistas de datos simplemente no podrían comparar grupos de datos 
eficientemente. 
Esta tarea además de importante es compleja y necesita de agrupaciones que se encarguen de trabajar en ella constantemente. Afortunadamente estas agrupaciones publican sus esfuerzos en plataformas 
como github para así poder aportar a la investigación, discutir soluciones, revisar problemas y atraer nuevos colaboradores.

\section{Librerías o frameworks para aplicaciones intensivas de frontend}

\section{Librerías para la visualización de datos}
Pero no cualquier librería 
Para la selección de librería se consideran los siguientes factores:

\begin{enumerate}
    \item {Debe ser un proyecto open source.}
    \item {De ser posible, debe tener bindings para ReactJS para facilidad en el desarrollo.}
    \item {Debe ser extensible para poder implementar aquellas visualizaciones que sean muy específicas.}
    \item {Deben ser longevas y tener cierta garantía de que será mantenida en el tiempo,
    asegurando así que para futuras mejoras de este TEG sigan disponibles.}
    \item {Debe ser fácil de utilizar}
\end{enumerate}

\subsection{ Data Driven Documents (D3) }
\begin{itemize}
    \item URL de repositorio \href{https://github.com/d3/d3}{https://github.com/d3/d3}
    \item Fecha de última versión: 11 de Abril de 2022
    \item Mantenida por la organización homónima
\end{itemize}

\subsection{ echarts }
\begin{itemize}
    \item URL de repositorio \href{https://github.com/apache/echarts}{https://github.com/apache/echarts}
    \item Fecha de última versión
    \item Mantenida por Apache
\end{itemize}

\subsection{ Comparación de bibliotecas }

Después de comparar estas librerías, sus APIs y leer otras investigaciones la decisión final es utilizar echarts

La base de esta conclusión es la simplicidad \cite{EchartsDecision}, mientras que d3 permite una gran flexibilidad, la curva de aprendizaje tiene una inclinación pronunciada. 
En comparación, echarts se ve como una herramienta plug and play que requiere mínima configuración y ahorra al desarrollador días de esfuerzo.

El tipo de gráficas a utilizar son las siguientes
\begin{itemize}    
    \item Número
    \item Línea
    \item Barra
    \item Área
    \item Torta
    \item Dispersión
    \item History flow
\end{itemize}

Para estas gráficas echarts provee soluciones pre-hechas y si es el caso de gráficos de interés en nichos específicos, como es el caso del History Flow, permite que fácilmente crees gráficas completamente nuevas. 
También provee gráficos de Sankey y Stacked Area Chart que son bastante parecidos y podrían ser personalizados para parecerse al History Flow. 

\section{ Proyectos alternos }
Estos dos gigantes en el mundo de la visualización de datos por si solos son opciones excelentes, pero trabajan sobre javascript vainilla y no están adaptadas para trabajar directamente sobre tecnologías como ReactJS. 
Para facilidad del desarrollo de la aplicación se utilizará una librería que se encargue de adaptar echarts a ReactJS.

\subsection{ echarts for react }
\begin{itemize}
    \item URL del repositorio \href{https://github.com/hustcc/echarts-for-react}{https://github.com/hustcc/echarts-for-react}
\end{itemize}