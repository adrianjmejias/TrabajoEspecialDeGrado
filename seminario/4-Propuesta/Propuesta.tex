


\section{Motivación e identificación del problema}

El éxito de Wikipedia como la mayor fuente información (en forma de artículos sobre eventos, personajes, definiciones, conceptos, locaciones y entre otros...) y su utilidad se debe a su extensa red de colaboradores.

Cada vez que estos ejecutan una acción en la Wikipedia -ya sea si aportan un artículo complejo o corrigen una tilde- dejan una huella llamada metadatos. Estos "rastros", conformados por fechas, número de líneas editadas, direcciones IP y demás, son de suma importancia para analistas de datos. Estos analistas, si tienen la experiencia, la práctica y tienen el tiempo para dedicarse a estudiar la API de Wikipedia, pueden tomar los metadatos, procesarlos y generar gráficos para buscar patrones de interés y sacarle el máximo rendimiento a esta función de la enciclopedia.

Puede tratarse de un proceso largo, y su complejidad limita quién puede formar parte. Si se tuviera una solución a esa dificultad, donde aquellos interesados "no técnicos" puedan aportar y donde los expertos (bien pueden ser los WikiWatchers que supervisan la calidad de la información) no tengan que dedicarle exhaustivas cantidades de tiempo para poder ayudar, podrían notarse cambios significativamente positivos en el crecimiento de la información, y discusión.

Es entendiendo estas razones que el presente seminario propone una herramienta que facilite esta tarea para todo tipo de usuarios, dándoles una interfaz fácil de usar y flexible para generar gráficos y métricas.

\section{Objetivos del trabajo}

\subsection{Objetivo General}
% TODO : Hacer Objetivo General
Crear una nueva versión del front-end de wikimetrics y extender con funcionalidades pertinentes para fomentar discusión sobre los artículos.

\subsection{Objetivos Específicos}
% TODO: Objectivos Específicos

\begin{itemize}
    \item Implementar una aplicación web adaptable que ofrezca las funcionalidades requeridas por el watcher respectivo de un wiki y que, además, pueda ser fácilmente reconocida por los motores de búsqueda a servicio de los usuarios.
    \item Consumir y extender la API de wikimetrics para desarrollar una aplicación web que habilite a sus usuarios construir y visualizar gráficas.
    \item Utilizar un método ágil para el desarrollo de la aplicación.
    \item Realizar el despliegue y puesta en producción de la aplicación.
\end{itemize}

\section{Estrategia de solución y método de desarrollo ágil a utilizar}

\subsection{Desarrollo Rápido de Aplicaciones (RAD)}

% TODO: revisar e iterar

El desarrollo rápido de aplicaciones es una metodología de desarrollo que prioriza la creación rápida, prototipos y la ágil retroalimentación sobre ciclos prolongados de desarrollo y prueba. Con el uso de RAD, los desarrolladores pueden realizar múltiples iteraciones y actualizaciones al software rápidamente sin la necesidad de iniciar un cronograma de desarrollo desde cero cada vez.

Esta metodología de desarrollo fue creada por James Martin en el año 1980 en IBM y fue formalizada con la publicación del libro \emph{Rapid Application Development} en 1991.

\begin{figure}[H]
    \centering
    \includegraphics[width=\textwidth]{Rapid-application-development.png}
    \caption{Fases del Desarrollo Rápido de Aplicaciones (enfoque de James Martin)}
    \label{fig:Rapid-application-development}
\end{figure}

\subsubsection{Fases del Desarrollo Rápido de Aplicaciones}

Según el enfoque de James Martin \cite{RADJamesMartin} el Desarrollo Rápido de Aplicaciones se divide en las siguientes fases:

\begin{itemize}
    \item \textbf{Fase de planificación:} durante esta etapa, los desarrolladores, clientes (usuarios de software) y miembros del equipo se comunican para definir, investigar y finalizar el alcance y los requisitos de su proyecto, así como los objetivos, las expectativas, los plazos y el presupuesto. 
    
    Uno de los beneficios del método RAD es que a pesar de haberse decidido los requisitos, estos pueden cambiar con facilidad en cualquier momento del ciclo de desarrollo.

    \item \textbf{Fase de diseño:} durante esta fase, los usuarios interactúan con analistas de sistemas y desarrollan modelos y prototipos que representan todos los procesos, entradas y salidas del sistema.
    
    Todos los errores y problemas se resuelven en un proceso iterativo. El desarrollador diseña un prototipo, el cliente (usuario) lo prueba y luego se reúnen para comunicar qué funcionó y qué no.

    \item \textbf{Fase de construcción:} esta etapa toma los prototipos y el sistema en su fase beta resultado de la etapa de diseño y lo convierte en un modelo funcional.

    Debido a que la mayoría de los problemas y cambios se abordaron durante la minuciosa fase de diseño iterativo, los desarrolladores pueden construir el modelo de trabajo final más rápidamente que siguiendo un enfoque tradicional de gestión de proyectos.

    \item \textbf{Fase de transición:} esta es la fase de implementación donde el producto terminado va al lanzamiento. Incluye conversión de datos, pruebas y cambio al nuevo sistema, así como capacitación de usuarios.

    Todos los cambios finales se realizan mientras los programadores y los clientes continúan buscando errores en el sistema.

\end{itemize}


\section{Trabajos relacionados}

Siendo Wikimedia y Wikipedia las más grandes comunidades y organizaciones dedicadas a recopilar datos, es lógico pensar que tiene consigo una abundante cantidad de seguidores capacitados y apasionados por aportar lo que puedan. A continuación se detallará el estado actual de las herramientas existentes para explorar la Wikipedia.

Estas herramientas cumplen diferentes objetivos:

\begin{enumerate}
    \item Identificar posibles candidatos para ser administradores.
    \item Identificar conflictos de edición.
    \item Proveer una forma visual de ver el desarrollo general de los artículos (crecimiento de artículo, detección de actividad maliciosa, entre otras cosas).
    \item Hacer uso de la información para proveer interfaces educativas tales como cronologías.
\end{enumerate}


\subsection{Toolforge}

Es una suite de herramientas estadísticas para las páginas, usuarios, wikis de Wikimedia. Entre las herramientas hosteadas más populares de toolforge se encuentran:

\begin{itemize}
    \item URL del proyecto \url{https://wikitech.wikimedia.org/wiki/Portal:Toolforge}
\end{itemize}

Comenzando con el proyecto madre de Wikimedia, este gigante sirve de plataforma para que colaboradores técnicos puedan usar servidores de Wikipedia para desarrollo y para poder levantar aplicaciones que mejoren la experiencia de todos los colaboradores de Wikipedia.
\\
Este proyecto es particularmente interesante porque podría ser un medio por el cual la aplicación WikiMetaView y sus dependencias podrían ser servidas al público.

\url{https://www.wikidata.org/wiki/Wikidata:Tools/Visualize_data}

\subsection{Sigma - Summary}

Esta herramienta creada por el usuario Sigma\(\Sigma\) permite obtener el historial de contribuciones de un usuario y filtrar las ediciones por una palabra clave. Tal como se puede observar en la imagen \ref{fig:sigma_summary}, el resultado de buscar el usuario \say{Clarityfiend} y filtrar por la palabra \say{space} es similar al que se puede encontrar en la lista de contribuciones por usuario de Wikipedia \cite{UserClarityfiend}, con el beneficio de poder filtrar por palabras claves.

\begin{itemize}
    \item URL del proyecto \url{https://sigma.toolforge.org/summary.py}
\end{itemize}

\begin{figure}[H]
    \centering
    \includegraphics[width=\textwidth]{proyectos-relacionados/sigma-summary.png}
    \caption{Sigma Summary}
    \label{fig:sigma_summary}
\end{figure}

\subsection{Histropedia Timeline}

\begin{itemize}
    \item URL del proyecto \url{http://histropedia.com/timeline/}.
\end{itemize}

Histropedia Timeline permite visualizar diferentes eventos en forma de línea de tiempo interactiva, usando el servicio de Wikidata para consultas Sparql \cite{WikidataSparql}. Es usada como herramienta educativa por distintas entidades como el Museo del Prado para que los usuarios puedan visualizar de forma sencilla el arte que se expone, tal como se puede observar en la imagen \ref{fig:museo-de-prado-timeline}. Además, la línea de tiempo permite asociar cada elemento de la colección con un evento histórico importante, de tal forma que se facilite el aprendizaje y se haga más dinámico.

\begin{figure}[H]
    \centering
    \includegraphics[width=1.0\textwidth]{proyectos-relacionados/museo-del-prado-timeline.png}
    \caption{Línea de tiempo que muestra la colección de arte del Museo del Prado}
    \label{fig:museo-de-prado-timeline}
\end{figure}



\begin{figure}[H]
    \centering
    \includegraphics[width=1.0\textwidth]{proyectos-relacionados/histropedia_timeline.png}
    \caption{Histropedia Timeline}
    \label{fig:histropedia_timeline}
\end{figure}


\url{https://en.wikipedia.org/wiki/Wikipedia:Tools}

\url{apersonbot.toolforge.org}
Este set de herramientas está enfocado en supervisar usuarios y sus contribuciones a la Wikipedia

- Candidate Search
- Aadminscore
- Articles for Creation Review History

\subsection{Xtools}
\begin{itemize}
    \item URL del proyecto \url{https://xtools.wmflabs.org/}
\end{itemize}

\subsection{Wiki Replay}
\begin{itemize}
    \item URL del proyecto \url{https://cosmiclattes.github.io/wikireplay/player}
\end{itemize}

Wiki Replay permite crear una línea de tiempo con los cambios en un artículo. Esta línea de tiempo
tiene el comportamiento de un reproductor de vídeo y permite que el usuario presione un botón de
reproducir y así ser guiado por como el contenido de un artículo ha ido evolucionando en el tiempo.

\begin{figure}[H]
    \centering
    \includegraphics[width=1.0\textwidth]{proyectos-relacionados/wikireplay.png}
    \caption{Interfaz de wikireplay}
    \label{wikireplay}
\end{figure}

\url{https://de.wikipedia.org/wiki/Benutzer:Atlasowa/edit_history_visualization}

\subsection{IBM History Flow tool}

\begin{itemize}
    \item URL del proyecto \url{http://alumni.media.mit.edu/~fviegas/papers/history_flow.pdf} \cite{HistoryFlowVisualizations}
\end{itemize}

History Flow es una herramienta de análisis de datos exploratorio que utiliza el registro de ediciones que posee Wikipedia para mostrar relaciones entre distintas versiones de un documento. El propósito de esta herramienta es mostrar tendencias o patrones generales del documento, pero al mismo tiempo preservar los detalles para una examinación cercana.

Como se puede observar en la imagen \ref{fig:history-flow-mechanism} cada barra vertical representa una versión del documento, siendo el largo proporcional a la cantidad de palabras que posee el documento. El color de la versión en toda su extensión permite asociar cada palabra con un usuario, identificando el autor de dicha versión. Por último, se utiliza la distancia entre versiones para separarlas de acuerdo a su fecha y así obtener información extra sobre la frecuencia de edición de cada palabra. Al realizar el procedimiento en la página \say{Brazil} \ref{fig:brazil-history-flow} se puede observar un evidente incremento en la longitud de las barras verticales, lo que implica un crecimiento drástico del contenido de la página.

\begin{figure}[H]
    \centering
    \includegraphics[width=\textwidth]{proyectos-relacionados/history-flow-mechanism.png}
    \caption{Explicación del mecanismo de visualización de History Flow \cite{HistoryFlowVisualizations}}
    \label{fig:history-flow-mechanism}
\end{figure}

\begin{figure}[H]
    \centering
    \includegraphics[width=\textwidth]{proyectos-relacionados/brazil-history-flow.png}
    \caption{La página "Brazil" mostrando un crecimiento abrupto y pocas contribuciones anónimas}
    \label{fig:brazil-history-flow}
\end{figure}

\subsection{Wiki History Flow}

\begin{itemize}
    \item URL del proyecto \url{https://github.com/rdmpage/wikihistoryflow}
\end{itemize}

Wiki History Flow es un proyecto que posee varios archivos de PHP que crean el history flow de una página en formato SVG. Este proyecto obtiene las revisiones directamente de Wikipedia haciendo uso de web scrapping y calculando la diferencia entre textos con el paquete de Pear Text\_Diff \cite{pear_text_diff}, para después gráficar las diferencias en forma de history flow.

\begin{figure}[H]
    \centering
    \includegraphics[width=\textwidth]{proyectos-relacionados/rdm-historyflow.png}
    \caption{History Flow creado por el proyecto}
    \label{fig:rdm-historyflow}
\end{figure}


\section{Herramientas}
En esta sección se enumerarán aquellas tecnologías a utilizarse en el desarrollo de la aplicación 

\subsection{Tecnologías para el ambiente de desarrollo}
\begin{itemize}
    \item Visual Studio Code como editor de código fuente.
    \item Git, como control de versiones.
    \item Navegadores web para hacer pruebas(Google Chrome, Firefox, etc).
    \item MongoDB como Base de Datos .
    \item Nodejs como servidor.
    \item Yan como manejador de paquetes de javascript.
\end{itemize}

Tecnologías o librerías web a utilizar:
\begin{itemize}
    \item Material UI.
    \item Next.js + ReactJS.
    \item Node.
    \item MongoDB.
    \item Fastify.
\end{itemize}

\section{Requerimientos}

    Para poder satisfacer las necesidades de los usuarios de la aplicación primero que hay que ponerse en sus zapatos para definir los problemas que tienen, explorar las posibles soluciones a estos problemas y priorizar para saber en qué invertir tiempo de desarrollo. Por ejemplo, es imperativo que los usuarios puedan solicitar gráficas de visualizaciones pero podrían prescindir de personalizar su perfil en la aplicación, e incluso el perfil en sí también podría ser obviado.

    Ahora, el hecho de que estas funcionalidades puedan omitirse no quiere decir que no enriquezcan la experiencia de usuario y en consecuencia aseguren el éxito de la aplicación, por lo que serán listadas de igual manera, refinadas e implementadas si así se decide.


\begin{enumerate}
    \item{La herramienta debe ser una aplicación web para la mayor facilidad de uso.}
    \item{La aplicación debe que estar alojada en un servidor, para poder ser accedida de manera remota y por cualquier usuario.}
    \item{La aplicación tiene que contar con una vista principal que incluya gráfica e información general de cada uno de los wikis del watchlist.}
    \item{Una página capaz de visualizar gráficas sobre propiedades de edición de un wiki.}
    \item{Los wikis a visualizar son extraídas del watchlist del usuario autenticado a través de Wikipedia (usando API de MediaWiki).}
    \item{Permitir editar las visualizaciones (alternar tipo de gráfica, cambiar los tipos de datos a visualizar).}
    \item{Los datos a representar en las visualizaciones son proporcionados a través del API Wikimetrics 2.0.}
    \item{Permitir al usuario crear una visualización nueva y guardarla de forma permanente, para este caso es necesario el uso de una base de datos, dado la poca cantidad de ediciones y la escala de datos que se puede llegar a manejar, es aconsejable inclinarse por una base de datos basada en  documentos}
\end{enumerate}

\section{Prototipo de interfaz}

\subsection{Página principal /home}
Para usuarios autenticados, les permite ver diferentes secciones relevantes para ellos como: 
\begin{enumerate}
    \item Watched: muestra una lista y un pequeño abstracto de los artículos que el usuario hace watching y tienen una visualización actualizada.
    \item Queue: muestra una lista de los artículos que el usuario solicitó para hacer una visualización pero que el server no ha podido solicitar o que todavía esta siendo procesado.
    \item Controversial: muestra una lista de aquellas visualizaciones que provocan discusión en la comunidad, estos pueden ser identificados tomando métricas como cantidad de comentarios, cantidad de ediciones en el artículo o cantidad de visualizaciones.
\end{enumerate}

\begin{figure}[H]
    \centering
    \includegraphics[width=1.0\textwidth]{prototipos/Index Auth.png}
    \caption{Prototipo de vista principal de un usuario autenticado}
    \label{PrototipoHomePage}
\end{figure}

Para usuarios no autenticados, se le redigirá al landing page (ruta /)

\subsection{Perfil de Watcher /watcher/:watcherId}
Para cualquier usuario muestra el perfil de un watcher, en él puedes ver todas las contribuciones de este junto con la lista de visualizaciones que ha creado.
Se permitirá un mínimo de personalización como una pequeña descripción sobre ellos mismos, un \textit{handle}\footnote{handle: } para poder referirse a ellos y se podrá enlazar el perfil de Wikipedia.

\begin{figure}[H]
    \centering
    \includegraphics[width=1.0\textwidth]{prototipos/Watcher Profile.png}
    \caption{Prototipo de página de perfil de watcher}
    \label{PrototipoWatchersProfile}
\end{figure}

\subsection{Landing page /}
TODO: poner imagen
Esta página se encarga de explicar las motivaciones de la App y dar una noción básica del funcionamiento para los watchers.
Tiene el objetivo de captar la atención 


\subsection{Página de configuración /settings}
TODO: Terminar seccion de settings
Acá el usuario puede bla bla

\subsection{Página de artículo /article}
\url{/article?title=<title>&url=<url>}
Muestra un artículo junto con su metadata y las visualizaciones creadas por los usuarios.

\begin{figure}[H]
    \centering
    \includegraphics[width=1.0\textwidth]{prototipos/Article Page.png}
    \caption{Prototipo de página de artículos}
    \label{PrototipoSettingsPage}
\end{figure}

\subsection{Sobre nosotros /about}
Breve resumen del proyecto, incluye el documento de tesis y documento de seminario así como datos de contacto y repositorios de GitHub para futuros contribuyentes.


\section{Ventanas Modales}
Las ventanas modales pueden sobreponerse a cualquier página en la aplicación y permiten al usuario efectuar una acción. Al cerrarse porque la acción fue ejecutada o el usuario decidió no hacer nada, se devuelve el foco del usuario mostrando a la página donde estaba.

Estas modales serán controladas por parámetros de consulta agregados al URL de cada página web. eg. si estamos en la página con la ruta \textbf{/watcher/adrian} y efectuamos alguna acción que requiera autenticación, la ruta cambiará a \textbf{/watcher/adrian?auth=true} para sugerir al usuario que debe ingresar con sus credenciales para poder ejecutar esa acción.

\subsection{Modal de autenticación}

Si este parámetro está en el URL. Presenta la modal de autenticación para que inicie sesión o se cree una cuenta de ser necesario.

\begin{figure}[H]
    \centering
    \includegraphics[width=1.0\textwidth]{prototipos/Modal-Auth.png}
    \caption{Modal de autenticación}
    \label{ModalAuth}
\end{figure}


\section{Planificación de las actividades}

TODO: UNA TABLA DE DOS COLUMNAS 
completar esta lista con candidatos a actividades que coincidan con iteraciones/sprints/... o como quiera que lo llamen dentro de su workflow.
ACTIVIDAD, TIEMPO ESTIMADO.

\begin{center}
\begin{tabular}{ | m{8cm} | m{5cm} | } 
 \hline
 Preparar el entorno de desarrollo & 1/2 semana \\ 
 \hline
 Estudiar API Wikimetrics 2.0 (Back-end) & 1/2 semana \\ 
 \hline
 Estudiar API MediaWiki & 1/2 semana \\ 
 \hline
 Integracion del API Wikimetrics 2.0 (Back-end) & 1/2 semana \\
 \hline
 Integracion del API MediaWiki & 1/2 semana \\
 \hline
\end{tabular}
\end{center}