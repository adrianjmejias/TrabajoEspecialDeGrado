
\section{Contexto}

% Resumen del estado actual de wikipedia. Situación alrededor la tesis tiene sentido
% Me va a situar en el problema que sirve de insumo para justificar el TEG

% DEBE identificar el problema, resuelto por el objetivo general

Con aproximadamente seis millones cuatrocientos mil artículos ( 6.400.000 ) la Wikipedia lidera como la enciclopedia mas extensa del mundo. No solo eso, sino que muy comúnmente al googlear sobre algún tema de interés siempre es el primer resultado.
Esto es gracias al esfuerzo colaborativo de miles de 




% Se hablan de soluciones previas y en eso introducimos wikimetrics y el front

Y son estos mismos grupos que han desarrollado soluciones para extender y analizar las wikis.


\subsection*{Herramientas de extension }
\begin{itemize}
    \item API Wikimetrics
    \item
\end{itemize}
% External tools

% presentamos nuestra solución

Nuestra labor entonces es facilitar este trabajo para ellos dejándoles crear sus propias gráficas sobre los artículos que deseen

\section{Objetivo general}
% TODO: refinar
Crear una nueva version del front-end de wikimetrics


\section{Objetivos específicos}
% TUTORIAL: Objetivos mas peque;os que conforman el todo
% También incluye peque;os estudios, decisiones y aprendizajes 

\begin{itemize}{}{}

    \item Implementar una aplicación web responsive que ofrezca las funcionalidades requeridas por un watcher de un wiki y que pueda ser reconocida por los motores de búsqueda.
    \item Consumir y extender la API de wikimetrics para desarrollar una aplicación web que habilite a sus usuarios construir y visualizar gráficas
    \item Definir los requerimientos de la aplicación
    \item Utilizar un método ágil para el desarrollo de la aplicación.
    \item Realizar el despliegue y puesta en producción de la aplicación

\end{itemize}


\section{Justificación}

Wikipedia contiene en si una masiva cantidad de datos "base" - como artículos, eventos, noticias, media y demás. Pero también ella misma genera nuevos datos con cada nueva adición y edición a su haber documental. Todos estos "rastros" que dejan miles de colaboradores dia a dia son conocidos como metadata.

Esta desde luego tiene un inmenso valor por si misma, y es refinada principalmente por analistas de datos y aficionados que quieren buscar patrones, relaciones o información que no es fácil o posible de distinguir solamente viendo números y fechas.

% TODO: Porque este trabajo es un trabajo util
% Aprovechar la data que genera wikipedia
% wikipedia tiene apis
% sacarle provecho
% apoyar a los watchers en su labor de vigilancia de sus artículos


Siendo wikimedia y wikipedia las mas grandes comunidades y organizaciones dedicadas a recopilar datos, es lógico pensar que tiene consigo una abundante cantidad de seguidores capacitados y apasionados por aportar lo que puedan. A continuacion se detallara el estado actual de las herramientas existentes para explorar la wikipedia.

Estas herramientas cumplen diferentes objetivos, 

\begin{enumerate}
    \item Identificar posibles candidatos para ser administrador
    \item Identifica 
    \item Etc.
  \end{enumerate}


https://wikitech.wikimedia.org/wiki/Portal:Toolforge
https://www.wikidata.org/wiki/Wikidata:Tools/Visualize_data

https://en.wikipedia.org/wiki/Wikipedia:Tools

apersonbot.toolforge.org

Este set de herramientas esta enfocado en supervisar usuarios y sus contribuciones a la wikipedia

- Candidate Search
- Aadminscore
- Articles for Creation Review History

https://xtools.wmflabs.org/


Sin embargo existe una oportunidad para aprovechar mejor la metadata de wikipedia, y explorar y explotar esta sera el problema a resolver de este trabajo especial de grado
https://sigma.toolforge.org/


https://interaction-timeline.toolforge.org/


https://cosmiclattes.github.io/wikireplay/player


https://de.wikipedia.org/wiki/Benutzer:Atlasowa/edit_history_visualization
\subsection{Herramientas}

\subsubsection*{Tecnologias para el ambiente de desarrollo}
\begin{itemize}
    \item Visual Studio Code.
    \item Git, como control de versiones.
    \item Navegadores web (Google Chrome)
    \item MongoDB
    \item 
\end{itemize}

\subsubsection*{Tecnologias usadas para desarrollar}
\begin{itemize}
    \item Visual Studio Code.
    \item Git, como control de versiones.
    \item Navegadores web (Google Chrome)
    \item MongoDB
    \item 
\end{itemize}



\begin{table}[]
\begin{tabular}{|l|l|}
\hline
Actividad & Tiempo Estimado \\ \hline
Prepararar entorno de desarrollo          &                 \\ \hline
Estudiar API existente de Wikimetrix            &                 \\ \hline
            &                 \\ \hline
\end{tabular}
\end{table}



Vistas


/watcher
Muestra el perfil de un watcher, en el puedes ver todas las contribuciones de este junto con la lista de visualizaciones que ha creado.
Se permitirá un mínimo de customización.


/home
Para usuarios autenticados, les permite ver diferentes secciones relevantes para ellos

1. Watched
    Muestra una lista y un pequeño abstracto de los articulos que el usuario hace watching y tienen una visualizacion actualizada

2. Waitlist
    Muestra una lista de los articulos que le usuario solicito para hacer una visualizacion pero que el server no ha podido solicitar

3. Explore
    3.1. Hot now
    Muestra una lista de aquellas visualizaciones que tienen mucha actividad 

    3.2. controversial
    Muestra una lista de aquellas visualizaciones que provocan discusion en la comunidad, caracterizado por muchos comentarios


/index
Esta pagina se encarga de explicar las motivaciones de la App y dar una nocion basica del funcionamiento para los watchers.
Tiene el objetivo de cap


/settings



/article
Muestra un articulo junto con su metadata y las visualizaciones creadas por los usuarios.



/about
Breve resumen del proyecto, incluye el documento de tesis y documento de seminario asi como datos de contacto y repositorios de github para futuros contribuyentes.

# Modales

Query Params Globales

?auth

Si este parámetro esta en el url. Presenta la modal de autenticacion 

?


\input{4-Propuesta/sections/requerimientos.tex}
La solucion plantada es un avance de la tesis por Leonardo Testa hecha en 20XX.


Actores


Watchers
Anonimos

\section{Distribución del documento}

% MID_TODO: Distribución del documento
