

\section{Motivación e identificación del problema}

% TODO : Hacer motivacion
Cada segundo ocurren




\section{Objetivos del trabajo}


\subsection{Objetivo General}
% TODO : Hacer Objetivo General
Crear una nueva version del front-end de wikimetrics y extender con funcionalidades pertinentes para fomentar discusión sobre los artículos

\subsection{Objetivos Específicos}
% TODO: Objectivos Específicos

\begin{itemize}
    \item Implementar una aplicación web responsive que ofrezca las funcionalidades requeridas por un watcher de un wiki y que pueda ser reconocida por los motores de búsqueda.
    \item Consumir y extender la API de wikimetrics para desarrollar una aplicación web que habilite a sus usuarios construir y visualizar gráficas
    \item Definir los requerimientos de la aplicación
    \item Utilizar un método ágil para el desarrollo de la aplicación.
    \item Realizar el despliegue y puesta en producción de la aplicación
\end{itemize}

\section{Estrategia de solución y método de desarrollo ágil a utilizar}

\subsection{Desarrollo Rápido de Aplicaciones (RAD)}

El desarrollo rápido de aplicaciones es una metodología de desarrollo que prioriza la creación rápida y prototipos y la retroalimentación rápida sobre ciclos prolongados de desarrollo y prueba. Con el uso de RAD, los desarrolladores pueden realizar multiples iteraciones y actualizaciones al software rápidamente sin la necesidad de iniciar un cronograma de desarrollo desde cero cada vez.

Esta metodología de desarrollo fue creada por James Martin en el año 1980 en IBM y fue formalizada con la publicación del libro \emph{Rapid Application Development} en 1991.

\begin{figure}[H]
    \includegraphics[scale=0.45]{Rapid-application-development.png}
    \caption{Fases del Desarrollo Rápido de Aplicaciones (enfoque de James Martin)}
    \label{fig:Rapid-application-development}
\end{figure}

\subsubsection{Fases del Desarrolo Rápido de Aplicaciones}

Según el enfoque de James Martin \cite{RADJamesMartin} el Desarrollo Rápido de Aplicaciones se divide en las siguientes fases:

\begin{itemize}
    \item \textbf{Fase de planificación:} Esta fase es equivalente a una reunión de alcance del proyecto. Aunque la fase de planificación se condensa en comparación con otras metodologías de gestión de proyectos.

          Durante esta etapa, los desarrolladores, los clientes (usuarios de software) y los miembros del equipo se comunican para determinar los objetivos y las expectativas del proyecto, así como los problemas actuales y potenciales que deben abordarse durante la construcción.

    \item \textbf{Fase de diseño:} Durante esta fase, los usuarios interactúan con analistas de sistemas y desarrollan modelos y prototipos que representan todos los procesos, entradas y salidas del sistema.

          Todos los errores y problemas se resuelven en un proceso iterativo. El desarrollador diseña un prototipo, el cliente (usuario) lo prueba y luego se reúnen para comunicar qué funcionó y qué no.

    \item \textbf{Fase de construcción:} Esta etapa toma los prototipos y el sistema en su fase beta resultado de la etapa de diseño y lo convierte en un modelo funcional.

          Debido a que la mayoría de los problemas y cambios se abordaron durante la minuciosa fase de diseño iterativo, los desarrolladores pueden construir el modelo de trabajo final más rápidamente que siguiendo un enfoque tradicional de gestión de proyectos.

    \item \textbf{Fase de transición:} Esta es la fase de implementación donde el producto terminado va al lanzamiento. Incluye conversión de datos, pruebas y cambio al nuevo sistema, así como capacitación de usuarios.

          Todos los cambios finales se realizan mientras los programadores y los clientes continúan buscando errores en el sistema.

\end{itemize}


\section{Trabajos similares, diferencias y ventajas de la solución a desarrollar}

Siendo Wikimedia y Wikipedia las mas grandes comunidades y organizaciones dedicadas a recopilar datos, es lógico pensar que tiene consigo una abundante cantidad de seguidores capacitados y apasionados por aportar lo que puedan. A continuación se detallará el estado actual de las herramientas existentes para explorar la Wikipedia.

Estas herramientas cumplen diferentes objetivos,

\begin{enumerate}
    \item Identificar posibles candidatos para ser administrador
    \item Identifica
    \item Etc.
\end{enumerate}


\subsection{Toolforge}

\begin{itemize}
    \item URL del proyecto \url{https://wikitech.wikimedia.org/wiki/Portal:Toolforge}
\end{itemize}

Comenzando con el proyecto madre de Wikimedia, este gigante sirve de plataforma para que colaboradores técnicos puedan usar servidores de Wikipedia para desarrollo y para poder levantar aplicaciones que mejoren la experiencia de todos los colaboradores de Wikipedia.
\\
Este proyecto es particularmente interesante porque podría ser un medio por el cual la aplicación WikiMetaView y sus dependencias podrían ser servidas al público.

\url{https://www.wikidata.org/wiki/Wikidata:Tools/Visualize_data}

\subsection{Sigma - ArticleInfo}
\begin{itemize}
    \item URL del proyecto \url{https://sigma.toolforge.org/articleinfo.py}
\end{itemize}

\begin{figure}[H]
    \centering
    \includegraphics[width=1.0\textwidth]{proyectos relacionados/sigma_article-info.png}
    \caption{Sigma Article Info}
    \label{sigma_articleInfo}
\end{figure}


% This isn't really useful lol
% \url{https://interaction-timeline.toolforge.org/}

\subsection{Histropedia Timeline}

\begin{itemize}
    \item URL del proyecto \url{http://histropedia.com/timeline/}
\end{itemize}

\begin{figure}[H]
    \centering
    \includegraphics[width=1.0\textwidth]{proyectos relacionados/histropedia_timeline.png}
    \caption{Histropedia Timeline}
    \label{histropedia_timeline}
\end{figure}


\url{https://en.wikipedia.org/wiki/Wikipedia:Tools}

\url{apersonbot.toolforge.org}
Este set de herramientas esta enfocado en supervisar usuarios y sus contribuciones a la Wikipedia

- Candidate Search
- Aadminscore
- Articles for Creation Review History

\subsection{Xtools}
\begin{itemize}
    \item URL del proyecto \url{https://xtools.wmflabs.org/}
\end{itemize}

\subsection{Wiki Replay}
\begin{itemize}
    \item URL del proyecto \url{https://cosmiclattes.github.io/wikireplay/player}
\end{itemize}

Se encarga de mostrar una linea de tiempo con los cambios en un articulo. Esta linea de tiempo
tiene el comportamiento de un reproductor de video y permite que el usuario presione un botón de
reproducir y así ser guiado por como el contenido de un artículo ha ido evolucionando en el tiempo.

\begin{figure}[H]
    \centering
    \includegraphics[width=1.0\textwidth]{proyectos relacionados/wikireplay.png}
    \caption{Interfaz de wikireplay}
    \label{wikireplay}
\end{figure}


\url{https://de.wikipedia.org/wiki/Benutzer:Atlasowa/edit_history_visualization}

\subsection{IBM History Flow tool}

\begin{itemize}
    \item URL del proyecto \url{http://alumni.media.mit.edu/~fviegas/papers/history_flow.pdf}
\end{itemize}

Es una herramienta de análisis de datos exploratorio, la cual utiliza el registro de ediciones que posee Wikipedia para mostrar relaciones entre distintas versiones de un documento. El proposito de esta herramienta es mostrar tendencias o patrones generales del documento, pero al mismo tiempo preservar los detalles para una examinación cercana.

Como se puede observar en la imagen \ref*{fig:history-flow-mechanism} cada barra vertical representa una version del documento, siendo el largo proporcional a la cantidad de palabras que posea el documento. El color de cada versión permite asociar cada palabra a un usuario en especifico, identificando a que autor pertenece en dicha versión. Por último se utiliza la distancia entre versiones para separarlas de acuerdo a su fecha y así obtener información extra sobre la frecuencia de edición de cada palabra. Al realizar el procedimiento en la página \say{Brazil} \ref*{fig:brazil-history-flow} se puede observar un evidente incremento en la longitud de las barras verticales, lo que implica un crecimiento drastico del contenido de la pagina.

\begin{figure}[H]
    \includegraphics[scale=0.5]{history-flow-mechanism.png}
    \caption{Explicación del mecanismo de visualización de History Flow}
    \label{fig:history-flow-mechanism}
\end{figure}

\begin{figure}[H]
    \includegraphics[scale=0.5]{brazil-history-flow.png}
    \caption{La pagina "Brazil" mostrando un crecimiento abrupto y pocas contribuciones anónimas}
    \label{fig:brazil-history-flow}
\end{figure}

\subsection{Wiki History Flow}

\begin{itemize}
    \item URL del proyecto \url{https://github.com/rdmpage/wikihistoryflow}
    \item
\end{itemize}

Este proyecto genera una gráfica de history flow dado el url de un articulo

\section{Herramientas}

\subsection{Tecnologías para el ambiente de desarrollo}
\begin{itemize}
    \item Visual Studio Code.
    \item Git, como control de versiones.
    \item Navegadores web (Google Chrome, Firefox, etc)
    \item Base de Datos (MongoDB)
    \item Servidor (Nodejs)
\end{itemize}

\subsection{Tecnologías o librerías web a utilizar}
\begin{itemize}
    \item Material UI
    \item Next.js + ReactJS
    \item Node
    \item MongoDB
    \item Fastify
\end{itemize}

\section{Requerimientos}
\begin{enumerate}
    \item{Una herramienta capaz de visualizar graficas sobre propiedades de edicion de un wiki.}
    \item{Los wikis a visualizar son extraídas del watchlist del usuario autenticado a traves de Wikipedia (usando API de MediaWiki).}
    \item{Permitir editar las visualizaciones (alternar tipo de grafica, cambiar los tipos de datos a visualizar).}
    \item{Los datos a representar en las visualizaciones son proporcionados a traves del API Wikimetrics 2.0.}
    \item{La herramienta tiene que ser una aplicacion web.}
    \item{La aplicacion tiene que contar con una vista principal que incluya grafica e informacion general de cada uno de los wiki del watchlist.}
    \item{Permitir al usuario crear una visualizacion nueva y guardarla de forma permanente, para este caso es necesario el uso de una base de datos (preferiblemente MongoDB)}
    \item{La aplicación tiene que estar alojada en un servidor, para poder ser accedida de manera remota.}
\end{enumerate}

\section{Prototipo de interfaz}

\subsection{Perfil de Watcher \textbackslash watcher\textbackslash:watcherId}
Para cualquier usuario muestra el perfil de un watcher, en él puedes ver todas las contribuciones de este junto con la lista de visualizaciones que ha creado.
Se permitirá un mínimo de personalización. Y se podrá enlazar el perfil de wikipedia

\begin{figure}[H]
    \centering
    \includegraphics[width=1.0\textwidth]{prototipos/Watcher Profile.png}
    \caption{Prototipo de página de perfil de watcher}
    \label{PrototipoWatchersProfile}
\end{figure}

\subsection{Página principal \textbackslash home}
Para usuarios autenticados, les permite ver diferentes secciones relevantes para ellos

\begin{figure}[H]
    \centering
    \includegraphics[width=1.0\textwidth]{prototipos/Index Auth.png}
    \caption{Prototipo de vista principal de un usuario autenticado}
    \label{PrototipoHomePage}
\end{figure}


Secciones del home
\begin{enumerate}
    \item Watched \\ Muestra una lista y un pequeño abstracto de los artículos que el usuario hace watching y tienen una visualización actualizada
    \item Queue \\ Muestra una lista de los artículos que le usuario solicito para hacer una visualización pero que el server no ha podido solicitar
    \item Controversial \\ Muestra una lista de aquellas visualizaciones que provocan discusión en la comunidad, caracterizado por muchos comentarios
\end{enumerate}

\subsection{Landing page \textbackslash}
TODO: poner imagen
Esta pagina se encarga de explicar las motivaciones de la App y dar una noción básica del funcionamiento para los watchers.
Tiene el objetivo de cap

\subsection{Página de configuración \textbackslash settings}
TODO: Terminar seccion de settings
Acá el usuario puede bla bla

\subsection{Página de artículo \textbackslash article}
\url{/article?title=<title>&url=<url>}
Muestra un articulo junto con su metadata y las visualizaciones creadas por los usuarios.d

\begin{figure}[H]
    \centering
    \includegraphics[width=1.0\textwidth]{prototipos/Article Page.png}
    \caption{Prototipo de página de artículos}
    \label{PrototipoSettingsPage}
\end{figure}

\subsection{Sobre nosotros \textbackslash about}
Breve resumen del proyecto, incluye el documento de tesis y documento de seminario asi como datos de contacto y repositorios de github para futuros contribuyentes.

\section{Modales}
Las modales pueden sobreponerse a cualquier página en la aplicación y permiten al usuario efectuar una acción.

Estas modales serán controladas por parámetros de consulta agregados al URL de cada página web
CleanArquitecture
\subsection{Modal de autenticación}

Si este parámetro esta en el url. Presenta la modal de autenticación

\begin{figure}[H]
    \centering
    \includegraphics[width=1.0\textwidth]{prototipos/Modal-Auth.png}
    \caption{Modal de autenticación}
    \label{ModalAuth}
\end{figure}
?auth=true

\section{Planificación de las actividades}

TODO: UNA TABLA DE DOS COLUMNAS

ACTIVIDAD, TIEMPO ESTIMADO.

\begin{lstlisting}
    Preparar el entorno de desarrollo 1/2 semana
    Estudiar API Wikimetrics 2.0 (Back-end) 1/2 semana
    Estudiar API MediaWiki 1/2 semana
    Integracion del API Wikimetrics 2.0 (Back-end) 1/2 semana
    Integracion del API MediaWiki 1/2 semana
\end{lstlisting}