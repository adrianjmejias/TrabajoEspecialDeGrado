% LATE_TODO: Introducción

% introducción: hablar un poco sobre wiki y la filosofía de wiki, también sobre la visualización de datos y terminamos con hablar sobre el trabajo en si

Un wiki es un sitio web que permite a sus usuarios colaborar en su estructura y contenido. Esta versatilidad que provee el concepto de Wiki es lo que lo convierte en una de las herramientas mas usadas en la actualidad para compartir información. La enciclopedia Wikipedia es el sitio web más popular basado en wiki.

El principal problema que maneja Wikipedia en cuanto a moderación de contenido viene como resultado de su propia filosofía "todos pueden editar", lo que conlleva a múltiples problemas tales como: vandalismo, escritura pobre, una mala estructura de página, peleas de edición, entre otras cosas. Por esta razón no existe una solución única para acabar con la existencia de \say{mal} contenido en Wikipedia, y es indispensable el uso de participación humana en procesos de moderación que implican complejos desafíos técnicos y éticos.

En la actualidad, gracias a la evolución del internet,
la información es considerada virtualmente ubicua y en constante cambio,
y lo que realmente ofrece valor es la capacidad individual de sintetizar esa información y relacionarla.
Como resultado de esto surge la filosofía wiki, en donde la información se comparte, y el conocimiento no se crea, sino se co-crea de forma colaborativa.

El concepto de la filosofía de wiki y el software utilizado para crear estos sitios web están intrínsecamente relacionados, y no se podría poner en práctica lo primero sin lo segundo. Esto es así debido a que el software debe proporcionar el medio para que pueda existir esa construcción colectiva de conocimiento, que es indispensable en la filosofía wiki.



