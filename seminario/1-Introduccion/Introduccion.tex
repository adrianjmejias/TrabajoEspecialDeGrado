% LATE_TODO: Introducción

% introducción: hablar un poco sobre wiki y la filosofía de wiki, también sobre la visualización de datos y terminamos con hablar sobre el trabajo en si

Si el internet es para todos y sin importar qué el mundo entero puede nutrirse de este, todos debemos colaborar. Cuando se crea el sistema Wiki por Ward Cunnigham en 1995, es bajo esta filosofía colaborativa y generosa. Se puede definir entonces el wiki como un sistema o herramienta cuyas páginas que lo utilicen permite a usuarios colaborar en su estructura y contenido. Esta versatilidad que provee el concepto de Wiki es lo que lo convierte en una de las herramientas más usadas y eficaz en la actualidad para compartir información. Bien podría servir de ejemplo La enciclopedia libre Wikipedia, que es es el sitio web que viene a la cabeza al hablar de wikis.

El principal problema que maneja Wikipedia en cuanto a moderación de contenido viene como resultado de su propia filosofía \say{todos pueden editar}, lo que conlleva a múltiples dificultades tales como: vandalismo, escritura pobre, una mala estructura de página, peleas de edición, entre otras cuestiones. Por esta razón no existe una solución única para acabar con la existencia de \say{mal} contenido en Wikipedia, y es indispensable el uso de participación humana en procesos de moderación que implican complejos desafíos técnicos y éticos.

Gracias a la evolución del internet en la actualidad se considera que la información es virtualmente ubicua y está en constante cambio, entonces lo que realmente ofrece valor es la capacidad individual de sintetizar esa información y relacionarla.

El concepto de la filosofía de wiki y el software utilizado para crear estos sitios web están intrínsecamente relacionados y no se podría poner en práctica lo primero sin lo segundo. Esto es así debido a que el software debe proporcionar el medio para que pueda existir esa construcción colectiva de conocimiento, que es indispensable en la filosofía wiki.



