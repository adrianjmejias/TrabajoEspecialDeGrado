% LATE_TODO: Introduccion

El término wiki proviene de la raíz Hawaiana \say{wiki}, que significa \say{rápido}, y fue propuesto por Ward Cunningham, quien a su vez define los sitios web wiki como "La base de datos más simple que puede existir" \cite{WhatIsWiki}. Con el tiempo este concepto fue evolucionando, y en la actualidad cuando hablamos de wiki nos referimos a un sitio web que permite a sus usuarios colaborar en su estructura y contenido. Estos sitios webs son impulsados por el motor wiki, también llamado MediaWiki, el cual es un Sistema Manejador de Contenido (CMS) que permite a los usuarios colaborar en el sitio web sin la necesidad de tener permisos de dueño o líder.

La enciclopedia Wikipedia es el sitio web más popular basado en wiki, que a su vez forma parte del movimiento Wikimedia, el cual incluye otros proyectos interrelacionados, tales como: Wiktionary, Wikiquote, Wikibooks, Wikisource, entre otros, cuyo propósito es usar el poder colaborativo de internet, y el concepto wiki, para compartir conocimiento gratuito de cualquier tipo.

En la actualidad, gracias a la evolución del internet, la información es considerada virtualmente ubicua y en constante cambio, y lo que realmente ofrece valor es la capacidad individual de sintetizar esa información y relacionarla. Como resultado de esto surge la filosofía wiki, en donde la información se comparte, y el conocimiento no se crea, sino se co-crea de forma colaborativa.

El concepto de la filosofía de wiki y el software utilizado para crear estos sitios web están intrínsecamente relacionados, y no se podría poner en práctica lo primero sin lo segundo. Esto es así debido a que el software debe proporcionar el medio para que pueda existir esa construcción colectiva de conocimiento, que es indispensable en la filosofía wiki.

Algunas de las características de software de los sitios web que hacen uso de esta filosofía son: 

El principal problema que maneja Wikipedia en cuanto a moderación de contenido viene como resultado de su propia filosofía "todos pueden editar", lo que conlleva a multiples problemas tales como: vandalismo, escritura pobre, una mala estructura de página, peleas de edición, entre otras cosas. Por esta razón no existe una solución única para acabar con la existencia de \say{mal} contenido en Wikipedia, y es indispensable el uso de participación humana en procesos de moderación que implican complejos desafíos técnicos y éticos.

Una de las formas que tiene Wikipedia de detectar vandalismo es usando las estadísticas de los artículos para verificar si hay una gran cantidad de ediciones de un articulo en un periodo muy corto de tiempo, o si estas revisiones vienen de la misma dirección IP, bloqueando o baneando las direcciones IP como método de reducción de vandalismo. Sin embargo, el bloqueo de IPs es en sí mismo un método que resulta contradictorio para el núcleo principal de la filosofía de wiki: "todos pueden editar", por lo que el bloqueo de IPs es utilizado siempre como última alternativa contra el vandalismo.


