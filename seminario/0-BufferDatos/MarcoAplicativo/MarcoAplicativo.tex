\section{Marco Aplicativo}

Revision de metodologias agiles vs pesadas, usos comunes y puntos a favor para seleccionar agile

agiles es mejor, comparemos entre ellas
\cite{ComparacionFrameworks}
\subsection{Revisión de metodologías ágiles para web }
% POST_SEMINARIO: Comparar y definir

\subsubsection{Lean}
\subsubsection{Scrum}
\subsubsection{TDD}
\subsubsection{RAD}

Requerimientos
watchpage/:watcherId: Home page debe ser personalizable por los watchers. (tambien debe ser un perfil publico)


Posibles adiciones

\begin{list}{}{}
    \item Dejar de manejar los usuarios internamente y manejar la autenticaci´on
          con el API de Wikipedia. Lo principal ser´ıa cumplir con los requerimientos que pide MediaWiki como API para crear la aplicaci´on usando
          OAuth2. De esta forma solo manejaremos un inicio de sesi´on usando el
          usuario de Wikipedia.
    \item Habilitar una opci´on para poder sincronizar los art´ıculos del watchlist
          del usuario de Wikipedia con los art´ıculos de esta plataforma.
          Analizar qu´e visualizaciones pueden ser frecuentes en los usuarios y
          ofrecerlas como visualizaciones predeterminadas.
    \item Implementar la gr´afica original de History Flow. Esta visualizaci´on requiere de mucho c´omputo, por lo que calcularlas en el front-end es
          inviable. La idea es delegarle esta funcionalidad a un servicio en el
          back-end que env´ıe los datos preparados para la visualizaci´on.
    \item Alguna funcionalidad para compartir dashboard de visualizaciones o
          visualizaciones individuales en modo de solo lectura
          79
    \item Implementar un servicio en el backend que se encargue de actualizar
          el estado de los art´ıculos extra´ıdos en el API de Usuarios. De esta
          manera, se evitan constantes peticiones HTTP en la aplicaci´on web
          (HTTP Polling).
\end{list}
