\section{Marco Aplicativo}

Revision de metodologías ágiles vs pesadas, usos comunes y puntos a favor para seleccionar agile

ágiles es mejor, comparemos entre ellas

\cite{ComparacionFrameworks}
\subsection{Revisión de metodologías ágiles para web }
% POST_SEMINARIO: Comparar y definir

\subsubsection{Lean}
\subsubsection{Scrum}
\subsubsection{TDD}
\subsubsection{RAD}

Requerimientos
watchpage/:watcherId: Home page debe ser personalizable por los watchers. (también debe ser un perfil publico)


Posibles adiciones

\begin{list}{}{}
    \item Dejar de manejar los usuarios internamente y manejar la autenticación
          con el API de Wikipedia. Lo principal sería cumplir con los requerimientos que pide MediaWiki como API para crear la aplicación usando
          OAuth2. De esta forma solo manejaremos un inicio de sesión usando el
          usuario de Wikipedia.
    \item Habilitar una opción para poder sincronizar los articulás del watchlist
          del usuario de Wikipedia con los artículos de esta plataforma.
          Analizar que visualizaciones pueden ser frecuentes en los usuarios y
          ofrecerlas como visualizaciones predeterminadas.
    \item Implementar la gráfica original de History Flow. Esta visualización requiere de mucho computo, por lo que calcularlas en el front-end es
          inviable. La idea es delegarle esta funcionalidad a un servicio en el
          back-end que envíe los datos preparados para la visualización.
    \item Alguna funcionalidad para compartir dashboard de visualizaciones o
          visualizaciones individuales en modo de solo lectura
          79
    \item Implementar un servicio en el backend que se encargue de actualizar
          el estado de los artículos extraídos en el API de Usuarios. De esta
          manera, se evitan constantes peticiones HTTP en la aplicación web
          (HTTP Polling).
\end{list}
