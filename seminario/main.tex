\documentclass[12pt]{report}
\usepackage[utf8]{inputenc}
\usepackage[spanish]{babel}
\usepackage[spanish=nohyphenation]{hyphsubst}
\usepackage{parskip} 
\usepackage{selinput}
\usepackage{tikz}
\usepackage{blindtext}
\usepackage{csquotes}
\usepackage{graphicx}
\usepackage[font=footnotesize,labelfont=bf,justification=centering,textfont=it]{caption}
\usepackage{float}
\usepackage{hyperref}
\usepackage{tabularx}
\usepackage{longtable}
\usepackage{listings}
\usepackage{color}
\usepackage{amsmath}
\usepackage{mathtools}
\usepackage{datetime}
\usepackage{listingsutf8}
\usepackage{dirtytalk}
\usepackage{mwe}
\usepackage[backend=biber]{biblatex}

\addbibresource{references.bib}
\bibliography{references.bib}
\makeatletter

\lstdefinelanguage{JavaScript}{
  keywords={typeof, new, true, false, catch, function, return, null, catch, switch, var, if, in, while, do, else, case, break},
  keywordstyle=\color{blue}\bfseries,
  ndkeywords={class, export, boolean, throw, implements, import, this},
  ndkeywordstyle=\color{darkgray}\bfseries,
  identifierstyle=\color{black},
  sensitive=false,
  comment=[l]{//},
  morecomment=[s]{/*}{*/},
  commentstyle=\color{purple}\ttfamily,
  stringstyle=\color{red}\ttfamily,
  morestring=[b]',
  morestring=[b]"
}

\lstset {
    language=JavaScript,           % language code
    basicstyle=\footnotesize,      % font size
    numbers=none,                  % where to put line numbers
    numberstyle=\footnotesize,     % numbers size
    numbersep=5pt,                 % how far the line numbers are from the code
    backgroundcolor=\color{white}, % background color
    showspaces=false,              % show spaces (with underscores)
    showstringspaces=false,        % underline spaces within strings
    showtabs=false,                % show tabs using underscores
    frame=single,                  % adds a frame around the code
    tabsize=4,                     % default tabsize
    breaklines=true,               % automatic line breaking
    columns=fullflexible,
    breakautoindent=false,
    framerule=1pt,
    xleftmargin=0pt,
    xrightmargin=0pt,
    breakindent=0pt,
    resetmargins=true
}

\setlength{\parindent}{15pt}
\setlength\bibitemsep{0.5\baselineskip}

\definecolor{lightgray}{rgb}{.9,.9,.9}
\definecolor{darkgray}{rgb}{.4,.4,.4}
\definecolor{purple}{rgb}{0.65, 0.12, 0.82}

\newdateformat{mydate}{\shortmonthname[\THEMONTH], \THEYEAR}

\renewcommand*{\thefootnote}{\arabic{footnote}}

\graphicspath{ {images/} }
\title{
	{Desarrollo de un editor de visualizaciones de propiedades de historiales de wikis}\\
	{\large Universidad Central de Venezuela}\\
}
\author{Adrian J. Mejias O. y Jose E. Tirado S.}



\addto\captionsspanish{
  \renewcommand{\listfigurename}
    {índice de figuras}
  \renewcommand{\listtablename}
    {índice de tablas}
  \renewcommand\lstlistingname{Código Fuente}
  \renewcommand\lstlistlistingname{índice de código fuente}
}
\renewcommand\spanishtablename{Tabla}
\newcommand{\tabitem}{~~\llap{\textbullet}~~}

\begin{document}

\begin{titlepage}
	\centering
	{\large República Bolivariana de Venezuela \par}
	{\large Universidad Central de Venezuela\par}
	{\large Facultad de Ciencias\par}
	{\large Escuela de Computación\par}\vspace{2cm}

	\includegraphics[scale=0.60]{ucv_logo.png}\par\vspace{1cm}
	{\scshape\Large\textbf{\@title} reescribir\par}
	\vfill

	{\large Tutor Prof. Eugenio Scalise \par}
	{\large \@author \par}
	{\large \mydate\today \par}
\end{titlepage}

\tableofcontents

\listoffigures

\chapter{Introducción}
% LATE_TODO: Introducción

% introducción: hablar un poco sobre wiki y la filosofía de wiki, también sobre la visualización de datos y terminamos con hablar sobre el trabajo en si

Si el internet es para todos y sin importar qué el mundo entero puede nutrirse de este, todos debemos colaborar. Cuando Ward Cunnigham creó el sistema Wiki en 1995, fue bajo esta filosofía colaborativa y generosa. Se puede definir entonces wiki como un sistema o herramienta que permite a usuarios colaborar en su estructura y contenido. Esta versatilidad que provee el concepto de wiki es lo que lo convierte en una de las herramientas más usadas y eficaces en la actualidad para compartir información. Bien podría servir de ejemplo la enciclopedia libre Wikipedia, que es el sitio web que viene a la cabeza al hablar de wikis.

El principal problema que maneja Wikipedia en cuanto a moderación de contenido viene como resultado de su propia filosofía \say{todos pueden editar}, lo que conlleva a múltiples dificultades tales como: vandalismo, escritura pobre, una mala estructura de página, peleas de edición, entre otras cuestiones. Por esta razón no existe una solución única para acabar con la existencia de \say{mal} contenido en Wikipedia, y es indispensable el uso de participación humana en procesos de moderación que implican complejos desafíos técnicos y éticos.

Por esta razón, en este trabajo se propone crear una herramienta que permita facilitar la moderación de contenido haciendo uso de visualizaciones, con el propósito de identificar patrones que conduzcan a actividades como el vandalismo o las peleas de edición, los cuales mediante investigaciones realizadas por IBM\cite{HistoryFlowVisualizations} se ha comprobado que muestran patrones visibles en gráficas especializadas.

\iffalse
Gracias a la evolución del internet en la actualidad se considera que la información es virtualmente ubicua y está en constante cambio, entonces lo que realmente ofrece valor es la capacidad individual de sintetizar esa información y relacionarla.

El concepto de la filosofía de wiki y el software utilizado para crear estos sitios web están intrínsecamente relacionados y no se podría poner en práctica lo primero sin lo segundo. Esto es así debido a que el software debe proporcionar el medio para que pueda existir esa construcción colectiva de conocimiento, que es indispensable en la filosofía wiki.
\fi






\chapter{Tecnologías para la visualización de datos en web}
Para el análisis de datos, el paso fundamental es la visualización de datos, sin este paso los analistas de datos simplemente no podrían comparar grupos de datos 
eficientemente. 
Esta tarea además de importante es compleja y necesita de agrupaciones que se encarguen de trabajar en ella constantemente. Afortunadamente estas agrupaciones publican sus esfuerzos en plataformas 
como github para así poder aportar a la investigación, discutir soluciones, revisar problemas y atraer nuevos colaboradores.

\section{Librerías o frameworks para aplicaciones intensivas de frontend}

\section{Librerías para la visualización de datos}
Pero no cualquier librería 
Para la selección de librería se consideran los siguientes factores:

\begin{enumerate}
    \item {Debe ser un proyecto open source.}
    \item {De ser posible, debe tener bindings para ReactJS para facilidad en el desarrollo.}
    \item {Debe ser extensible para poder implementar aquellas visualizaciones que sean muy específicas.}
    \item {Deben ser longevas y tener cierta garantía de que será mantenida en el tiempo,
    asegurando así que para futuras mejoras de este TEG sigan disponibles.}
    \item {Debe ser fácil de utilizar}
\end{enumerate}

\subsection{ Data Driven Documents (D3) }
\begin{itemize}
    \item URL de repositorio \href{https://github.com/d3/d3}{https://github.com/d3/d3}
    \item Fecha de última versión: 11 de Abril de 2022
    \item Mantenida por la organización homónima
\end{itemize}

\subsection{ echarts }
\begin{itemize}
    \item URL de repositorio \href{https://github.com/apache/echarts}{https://github.com/apache/echarts}
    \item Fecha de última versión
    \item Mantenida por Apache
\end{itemize}

\subsection{ Comparación de bibliotecas }

Después de comparar estas librerías, sus APIs y leer otras investigaciones la decisión final es utilizar echarts

La base de esta conclusión es la simplicidad \cite{EchartsDecision}, mientras que d3 permite una gran flexibilidad, la curva de aprendizaje tiene una inclinación pronunciada. 
En comparación, echarts se ve como una herramienta plug and play que requiere mínima configuración y ahorra al desarrollador días de esfuerzo.

El tipo de gráficas a utilizar son las siguientes
\begin{itemize}    
    \item Número
    \item Línea
    \item Barra
    \item Área
    \item Torta
    \item Dispersión
    \item History flow
\end{itemize}

Para estas gráficas echarts provee soluciones pre-hechas y si es el caso de gráficos de interés en nichos específicos, como es el caso del History Flow, permite que fácilmente crees gráficas completamente nuevas. 
También provee gráficos de Sankey y Stacked Area Chart que son bastante parecidos y podrían ser personalizados para parecerse al History Flow. 

\section{ Proyectos alternos }
Estos dos gigantes en el mundo de la visualización de datos por si solos son opciones excelentes, pero trabajan sobre javascript vainilla y no están adaptadas para trabajar directamente sobre tecnologías como ReactJS. 
Para facilidad del desarrollo de la aplicación se utilizará una librería que se encargue de adaptar echarts a ReactJS.

\subsection{ echarts for react }
\begin{itemize}
    \item URL del repositorio \href{https://github.com/hustcc/echarts-for-react}{https://github.com/hustcc/echarts-for-react}
\end{itemize}

\chapter{Tecnologías para el desarrollo web}

TODO mejorar redaccion de esta intro

Así como el propósito de un libro es ser leído y su éxito se mide en la cantidad de personas que lo lean, el éxito de un sitio web informativo se mide en su crecimiento; vive para tener una relación simbiótica con los usuarios que proveen el contenido y con él atraen lectores que se convierten en editores y en nuevos creadores. El sitio web por su parte, aporta todas las herramientas necesarias para que todos puedan participar.

Con el tiempo, y si los usuarios tienen una experiencia positiva este "libro" web, se puede esperar que esté convertido en ecosistema de datos. Para garantizar una buena experiencia de usuario Petter Morville \cite{UXFactors} propone tener en cuenta siguientes factores:

\begin{itemize}
  \item Utilidad: el producto realizado debe ser de utilidad para el público objetivo.
  \item Usabilidad: debe ser un sistema que sea simple y fácil de usar para el usuario.
  \item Deseabilidad: cuando un producto es deseable, los usuarios se sienten atraídos a él.
  \item Encontrable: se debe construir un sistema que sea navegable e instintivo al recorrer, de tal forma que los usuarios puedan encontrar lo que necesiten fácilmente.
  \item Accesibilidad: el sistema debe ser fácilemente usado por la mayor cantidad de personas posibles.
  \item Credibilidad: el sistema debe proporcionar seguridad al usuario acerca de la veracidad de su contenido.
  \item Valor: una experiencia de usuario valiosa implica un buen uso de los demás factores.
\end{itemize}

TODO revisar

Cumpliendo con las características mencionadas se puede apostar por el éxito de la web, pero para dar el salto de la teoría a la práctica hay que realmente entender la buena experiencia del usuario. Razón por la cual se desglosarán los factores en el presente capítulo.

\section{Arquitectura}
TODO Eugenio propone cambiar el nombre del capítulo

TODO Explicar como el SEO tiene que ver con la arquitectura

TODO revisar

Para que una aplicación sea encontrable — y así usada — por internautas es fundamental que tenga una buena relación con los motores de búsqueda.

Sin embargo también para asegurar la larga vida y mantenibilidad de la aplicación y la facilidad de desarrollo se debe tomar en cuenta herramientas extensamente empleadas contemporáneamente como Angular, React y Vue.

El problema entonces recae en que estas tecnologías por si solas son meramente para aplicaciones de una sola página. Los motores de búsqueda hacen lo que pueden pero el HTML estático manejado por las SPA \footnote{\textit{Single Page Application}: es un tipo de aplicación que ejecuta todo su contenido en una sola página.} es mínimo, resultando en que los motores de búsqueda no puedan inferir de qué trata la página, afectando su habilidad de ser encontrada e indexada.

Como remedio surge un nuevo paradigma con el nombre de Server Side Rendering o simplemente SSR \cite{ServerSideRendering}; donde se utilizan  tecnologías SPA como motor de plantillas, para así generar HTML digerible por los motores de búsqueda. Una vez el HTML estático llega al cliente, este atraviesa un proceso conocido como \textit{hydration} \footnote{\textit{Hydration}: es una técnica en la que se convierte una página de HTML estático en una página dinámica con el uso de eventos JavaScript.}, donde retoma sus funcionalidades dinámicas características de SPA.

Así se logra el perfecto balance de herramientas actuales y fáciles de usar, que también cumplen con los requerimientos de los motores de búsqueda para indexar páginas web.

\section{Tecnologías para el desarrollo}

Actualmente existe un catálogo muy amplio de tecnologías de desarrollo web, por lo que escoger las debidas herramientas y analizar sus compatibilidades es clave para obtener el mejor conjunto de herramientas.

En esta sección se tratarán de las herramientas de desarrollo web usadas para llevar a cabo el proyecto, y se descompondrán para su fácil entendimiento en 2 categorías: Servidor y Cliente.

\section{Tecnologías del lado del Servidor}

Como bien sabemos la página web debe mostrar datos de artículos y watchers, y otras funcionalidades como autenticación y guardar visualizaciones, que dependen de un servidor, esta sección se encargará de estudiar y decidir las tecnologías a utilizar para este.

\subsection{Fastify}

Fastify es un framework web para NodeJS de código abierto concentrado en proporcionar mejor rendimiento, y una arquitectura flexible. Proporciona un ecosistema de componentes.

TODO explicar qué hace Fastify

Las principales características de Fastify son \cite{FastifyCoreFeatures}:

\begin{itemize}
    \item Alto rendimiento: si se compara la velocidad de Fastify con otros frameworks web como Express, tal como se aprecia en la figura \ref{fig:fastify_vs_express}, notamos que Fastify es aproximadamente un 20\% más rápido que Express.
    \item Extensible: Fastify es completamente extensible mediante el uso de \textit{hooks} \footnote{\textit{Hook}: es una técnica que permite interceptar llamados a funciones con el propósito de alterar el comportamiento de la aplicación a eventos específicos.}, plugins \footnote{\textit{Plugin}: en el contexto de Fastify, un plugin permite agregar a funcionalidades extra (rutas, decoradores entre otras cosas) para que puedan ser usadas por el framework.} o decoradores \footnote{\textit{Decoradores}: permite agregar funcionalidades al núcleo de Fastify o a las peticiones HTTP usadas, durante su cíclo de vida.}.
    \item Basado en esquemas: a pesar de no ser obligatorio, es recomendado usar el esquema JSON para validar las rutas y serializar las salidas, internamente Fastify compila el esquema en una función de alto rendimiento.
    \item Registro: usualmente el guardar un registro puede ser costoso, por esa razón Fastify hace uso de un logger eficiente que casi remueve ese costo.
    \item Amigable para los desarrolladores: este framework está desarrollado para ser bastante expresivo y ayudar a los desarrolladores en su uso, sin sacrificar rendimiento o seguridad.
    \item Listo para usar con Typescript: la instalación inicial del framework provee compatibilidad predeterminada con Typescript.
\end{itemize}


\begin{figure}[H]
    \centering
    \includegraphics[width=\textwidth]{fastify_vs_express.png}
    \caption{ Número de peticiones por segundo para distinta cantidad de conexiones\cite{FastifyVsExpress}}
    \label{fig:fastify_vs_express}
\end{figure}

\subsection{MongoDB}

MongoDB es un sistema manejador de base de datos de documentos NoSQL escalable y flexible diseñado para lidiar con los conflictos que poseen las bases de datos relacionales y las limitaciones de otras soluciones NoSQL. En lugar de guardar los datos en tablas, tal y como se hace en las bases de datos relacionales, MongoDB guarda estructuras de datos BSON (una especificación similar a JSON) con un esquema dinámico, lo que facilita la integración de los datos.

No tener restricciones al estructurar los datos facilita el desarrollo, siempre y cuando el desarrollador sea consciente de la redundancia que puede causar en la base de datos. Los documentos BSON son almacenados en colecciones dentro de MongoDB y estos son consultados y mutados usando un estandard JSON, no un lenguaje de consulta como SQL.

Como se puede observar en la imagen \ref{fig:ranking-db}, mongoDB es actualmente la opción más usada entre las bases de datos NoSQL, ocupando el puesto número 4 en el ranking de bases de datos propuesto por Stack Overflow \cite{StackOverflowSurvey}. Esto proporciona suficiente certeza de que MongoDB será mantenido por años en el futuro.

\begin{figure}[H]
    \centering
    \includegraphics[width=\textwidth]{ranking-db.png}
    \caption{Ranking de bases de datos más populares de 2022}
    \label{fig:ranking-db}
\end{figure}

Algunas de las características más importantes de MongoDB son:

\begin{itemize}
  \item Consultas ad-hoc optimizadas para análisis en tiempo real.
  \item Indexación adecuada para mejores tiempos de ejecución de consultas.
  \item Replicación para una mejor disponibilidad y estabilidad de los datos.
  \item \textit{Sharding}\footnote{\textit{Sharding}: proceso de distribuir un conjunto de datos en múltiples bases de datos, que después pueden ser almacenados en múltiples ordenadores.}.
  \item Balanceo de carga.
\end{itemize}

Para realizar operaciones sobre la base de datos MongoDB usando un servidor NodeJS es necesario usar un MongoDB Driver compatible con NodeJS, el cual permite establecer una conexión con la base de datos, y que provee una API que facilita la interacción con la misma. A continuación se mostrará un código básico de conexión y consulta de datos usando el NodeJS MongoDB Driver.

TODO buscar como ponerle una fuente distinta a esto (pref monospace)
\begin{lstlisting}
    import { MongoClient } from "mongodb";
    const uri = "<connection string uri>";
    const client = new MongoClient(uri);

    async function run() {
        await client.connect();
        const database = client.db("sample_mflix");
        const movies = database.collection("movies");
        const query = { title: "The Room" };
        const options = {};
        const movie = await movies.findOne(query, options);
    }
\end{lstlisting}

En el código se puede observar cómo se utiliza el cliente de MongoDB para conectarse a la base de datos \textquote{sample\_mflix} y después consultar todas las películas cuyo título sea igual a \textquote{The Room}.

\section{Tecnologías del lado del Cliente}

Se entienden como tecnologás del lado del cliente la programación con la que interactuará el usuario, es imperativo que 
TODO por qué escogimos las que escogimos?

\subsection{ReactJS}

ReactJS es una librería de JavaScript de código abierto desarrollada por Facebook para facilitar la creación de componentes interactivos y reutilizables, para desarrollo de interfaces de usuario, especialmente aplicaciones de una sola página.

React maneja el concepto de \say{programación reactiva} \cite{ReactiveProgramming} haciendo uso de un DOM Virtual, lo que le permite determinar qué partes del DOM han cambiado comparando contenidos entre la versión nueva y la almacenada en el DOM virtual, para así propagar los datos generando cambios en la aplicación, es decir, la aplicación \say {reacciona} a cambios en los datos ejecutando una serie de eventos y re-renderizando solo aquellos componentes que lo ameriten (característica que lo hace altamente eficiente). 

Otras características que destacan en React se explican a continuación.

    \noindent \textbf{Componentes} \hfill

        El código de React es hecho con entidades llamadas componentes. Los componentes pueden ser renderizados en elementos particulares del DOM usando la librería de React DOM. Estos componentes son capaces de recibir parámetros conocidos como "propiedades del componente" de la siguiente forma: \hfill 

        \begin{lstlisting}
            ReactDOM.render(<Greeter greeting="Hello World!" />, document.getElementById('myReactApp'));
        \end{lstlisting}

        React ofrece 2 formas de declarar componentes: mediante el uso de funciones o clases. Generalmente se usa una de las dos opciones de forma situacional.

   \noindent \textbf{JSX} \hfill

        JSX, también llamado Javascript XML, es una extensión a la sintaxis del lenguaje JavaScript. Este provee una forma de estructurar componentes usando una sintaxis familiar para muchos desarrolladores. Los componentes de React son usualmente escritos en JSX, aunque también pueden ser escritos usando JavaScript puro.

        Un ejemplo de código JSX:

        \begin{lstlisting}
            class App extends React.Component {
                render() {
                    return (
                    <div>
                        <p>Header</p>
                        <p>Content</p>
                        <p>Footer</p>
                    </div>
                    );
                }
            }
        \end{lstlisting}

   \noindent \textbf{Hooks} \hfill 

        Los hooks son funciones que permiten a los desarrolladores \say{engancharse} a los estados de React y a ciertos puntos dentro del ciclo de vida de los componentes.

        React proporciona algunos hooks integrados tales como: useState, useContext, useReducer, useMemo y useEffect son los más usados y permiten controlar los estados y manejar eventos.


\subsection{Next.js}

Next.js es un framework desarrollado sobre NodeJS que permite a las aplicaciones de React usar funcionalidades como el renderizado del lado servidor y la generación de páginas web estáticas.

Por defecto, Next.js pre-renderiza cada página. Esto significa que Next.js genera HTML para cada página en adelanto, en vez de hacerse con Javascript del lado del cliente.  El pre-renderizado puede resultar en mejor rendimiento y SEO.

Cada HTML generado es asociado con el mínimo código JavaScript necesario para que funcione la página. Cuando una página es cargada en el explorador, su código JavaScript se ejecuta y hace la página totalmente interactiva. A este proceso de le conoce como \textit{hydration}

Next.js ofrece 2 formas de pre-renderizado:

\begin{itemize}
  \item Generación estática: el HTML es pre-generado y puede ser servido estáticamente.
  \item Renderizado del lado servidor: el HTML es generado al recibir una petición del cliente, necesita un servidor que procese las peticiones.
\end{itemize}

\subsection{Material UI}

Material UI es un framework de React creado con el objetivo de proporcionar una forma sencilla a los desarrolladores de aplicar los principios de Material Design \cite{MaterialDesignPrinciples} usando componentes de React. Este framework se destaca por la gran variedad de componentes disponibles, que van desde elementos visuales sencillos como botones o selectores, hasta complejos como modales o drawers. Además de su variedad de componentes, Material UI también destaca por su gran cantidad de contribuyentes, lo que asegura la longevidad y mantenibilidad del framework.

Entre las principales características que posee Material UI se encuentran:

\begin{itemize}
  \item Posibilidad de crear temas, lo que permite personalizar los estilos predeterminados de los componentes de la librería de forma global, o acotada.
  \item Fue construido usando la estrategia Mobile First \cite{MobileFirst}, en la cual se crea el código primero para dispositivos móviles, y después se escalan usando las reglas de CSS pertinentes.
  \item Los componentes de Material UI se consideran autosuficientes, por lo que solo utilizan los estilos CSS necesarios para mostrar el componente en pantalla, es decir, no dependen de hojas de estilos globales como en el caso de normalize.css o bootstrap.
\end{itemize}

A continuación se mostrará el uso de algunos componentes que ofrece la librería:

\begin{itemize}
  \item Botones: Material UI viene incluido con 3 variantes visuales de botones, los cuales son usados dependiendo de la importancia o énfasis que se le quiera dar al botón. Además, también posee variaciones de color y tamaño predeterminados y la posibilidad de usar íconos que funcionan como botones.

        \begin{lstlisting}
                <Button variant="text">Text</Button>
                <Button variant="contained">Contained</Button>
                <Button variant="outlined">Outlined</Button>
                <Button color="secondary">Secondary</Button>
                <Button size="small">Small</Button>
                <IconButton aria-label="delete">
                    <DeleteIcon />
                </IconButton>
            \end{lstlisting}



  \item Selectores: el selector comúnmente se encuentra envuelto en un componente de FormControl, el cual provee contexto para hacer uso de campos requeridos, manejo de errores y uso de propiedades como ``seleccionado" o ``campo lleno".  TODO ACOMODAR O

        \begin{lstlisting}
                <FormControl fullWidth>
                    <InputLabel id="demo-simple-select-label">Age</InputLabel>
                    <Select
                        labelId="demo-simple-select-label"
                        id="demo-simple-select"
                        value={age}
                        label="Age"
                        onChange={handleChange}
                    >
                        <MenuItem value={10}>Ten</MenuItem>
                        <MenuItem value={20}>Twenty</MenuItem>
                        <MenuItem value={30}>Thirty</MenuItem>
                    </Select>
                </FormControl>
            \end{lstlisting}



  \item Sliders: Material UI permite personalizar los Sliders de múltiples maneras, tales como usar valores discretos, cambiar el tamaño del Slider, o utilizar íconos personalizados en los extremos o en la perilla de este, lo que contextualiza al usuario sobre lo que representa el valor del Slider.

        \begin{lstlisting}
                const marks = [
                    {
                        value: 0,
                        label: "0km"
                    },
                    {
                        value: 20,
                        label: "20km"
                    },
                    {
                        value: 37,
                        label: "37km"
                    },
                    {
                        value: 100,
                        label: "100km"
                    },
                ];
                <Slider aria-label="Volume" value={25} onChange={handleChange} />
                <Slider defaultValue={30} step={10} marks min={10} max={110} disabled />
                <Slider
                    aria-label="Custom marks"
                    defaultValue={20}
                    getAriaValueText={valuetext}
                    step={10}
                    valueLabelDisplay="auto"
                    marks={marks}
                />
            \end{lstlisting}

\end{itemize}



\chapter{Propuesta de Trabajo Especial de Grado}


\section{Motivación e identificación del problema}

\section{Objetivos del trabajo}

\subsection{Objetivo General}

% TODO: refinar
Crear una nueva version del front-end de wikimetrics y extender con funcionalidades pertinentes para fomentar discusión sobre los artículos

\subsection{Objetivos Específicos}

% TUTORIAL: Objetivos mas peque;os que conforman el todo
% También incluye peque;os estudios, decisiones y aprendizajes 

\begin{itemize}
    \item Implementar una aplicación web responsive que ofrezca las funcionalidades requeridas por un watcher de un wiki y que pueda ser reconocida por los motores de búsqueda.
    \item Consumir y extender la API de wikimetrics para desarrollar una aplicación web que habilite a sus usuarios construir y visualizar gráficas
    \item Definir los requerimientos de la aplicación
    \item Utilizar un método ágil para el desarrollo de la aplicación.
    \item Realizar el despliegue y puesta en producción de la aplicación
\end{itemize}

\section{Estrategia de solución y método de desarrollo ágil a utilizar}

TODO: DECIDIR ENTRE

\begin{itemize}
    \item Kanban: Tiene como objetivo la mejora continua, la flexibilidad en la gestión de tareas y un flujo de trabajo mejorado. Con este enfoque ilustrativo, el progreso de todo el proyecto se puede comprender fácilmente de un vistazo. Para esto hace uso del tablero Kanban, que es una herramienta que visualiza todo el proyecto para rastrear el flujo de su proyecto. A través de este enfoque gráfico de los tableros Kanban, un miembro nuevo o una entidad externa puede comprender lo que está sucediendo en este momento, las tareas completadas y las tareas futuras.
    \item Rapid application development (RAD):  es una forma de metodología de desarrollo de software ágil que prioriza las versiones e iteraciones rápidas de prototipos. A diferencia del método Waterfall, RAD enfatiza el uso de software y los comentarios de los usuarios sobre la planificación estricta y el registro de requisitos.
\end{itemize}

TODO: ELABORAR Y PONER DIBUJITOS ILUSTRATIVOS


\section{Trabajos similares, diferencias y ventajas de la solución a desarrollar}
Siendo wikimedia y wikipedia las mas grandes comunidades y organizaciones
dedicadas a recopilar datos,
es lógico pensar que tiene consigo una abundante cantidad de seguidores capacitados y
apasionados por aportar lo que puedan. A continuación se detallara el estado actual de las
herramientas existentes para explorar la wikipedia.

Estas herramientas cumplen diferentes objetivos,

\begin{enumerate}
    \item Identificar posibles candidatos para ser administrador
    \item Identifica
    \item Etc.
\end{enumerate}


\url{https://wikitech.wikimedia.org/wiki/Portal:Toolforge}
Comenzando con el proyecto madre de wikimedia, este gigante sirve de plataforma para que colaboradores técnicos puedan usar servidores de wikipedia para desarrollo y para poder levantar aplicaciones que mejoren la experiencia de todos los colaboradores de wikipedia.
\\
Este proyecto es particularmente interesante porque podría ser un medio por el cual la aplicación WikiMetaView y sus dependencias podrían ser servidas al público.


\url{https://www.wikidata.org/wiki/Wikidata:Tools/Visualize_data}
\url{https://en.wikipedia.org/wiki/Wikipedia:Tools}

\url{apersonbot.toolforge.org}
Este set de herramientas esta enfocado en supervisar usuarios y sus contribuciones a la wikipedia

- Candidate Search
- Aadminscore
- Articles for Creation Review History

\section{xtools}
\begin{itemize}
    \item URL del proyecto \url{https://xtools.wmflabs.org/}
\end{itemize}


\url{https://sigma.toolforge.org/}


\url{https://interaction-timeline.toolforge.org/}


\subsection{wikireplay}
\begin{itemize}
    \item URL del proyecto \url{https://cosmiclattes.github.io/wikireplay/player}
\end{itemize}

Se encarga de mostrar una linea de tiempo con los cambios en un articulo. Esta linea de tiempo
tiene el comportamiento de un reproductor de video y permite que el usuario presione un botón de
reproducir y así ser guiado por como el contenido de un artículo ha ido evolucionando en el tiempo.

\begin{figure}[]
    \centering
    \includegraphics[width=1.0\textwidth]{proyectos relacionados/wikireplay.png}
    \caption{Interfaz de wikireplay}
    \label{wikireplay}
\end{figure}


\url{https://de.wikipedia.org/wiki/Benutzer:Atlasowa/edit_history_visualization}


\subsection{wikihistoryflow}

\begin{itemize}
    \item URL del proyecto \url{https://github.com/rdmpage/wikihistoryflow}
    \item
\end{itemize}

Este proyecto genera una gráfica de history flow dado el url de un articulo

\section{Herramientas}

\subsection{Tecnologías para el ambiente de desarrollo}
\begin{itemize}
    \item Visual Studio Code.
    \item Git, como control de versiones.
    \item Navegadores web (Google Chrome, Firefox, etc)
    \item MongoDB
    \item
\end{itemize}

\subsection{Tecnologías o librerías web a utilizar}
\begin{itemize}
    \item Material UI
    \item Next.js + ReactJS
    \item Node
    \item MongoDB
    \item Fastify
\end{itemize}

\section{Requerimientos}
\begin{enumerate}
    \item{Una herramienta capaz de visualizar graficas sobre propiedades de edicion de un wiki.}
    \item{Los wikis a visualizar son extraídas del watchlist del usuario autenticado a traves de Wikipedia (usando API de MediaWiki).}
    \item{Permitir editar las visualizaciones (alternar tipo de grafica, cambiar los tipos de datos a visualizar).}
    \item{Los datos a representar en las visualizaciones son proporcionados a traves del API Wikimetrics 2.0.}
    \item{La herramienta tiene que ser una aplicacion web.}
    \item{La aplicacion tiene que contar con una vista principal que incluya grafica e informacion general de cada uno de los wiki del watchlist.}
    \item{Permitir al usuario crear una visualizacion nueva y guardarla de forma permanente, para este caso es necesario el uso de una base de datos (preferiblemente MongoDB)}
    \item{La aplicación tiene que estar alojada en un servidor, para poder ser accedida de manera remota.}
\end{enumerate}

\section{Prototipo de interfaz}

\subsection{Perfil de Watcher \textbackslash watcher\textbackslash:watcherId}
Para cualquier usuario muestra el perfil de un watcher, en él puedes ver todas las contribuciones de este junto con la lista de visualizaciones que ha creado.
Se permitirá un mínimo de personalización. Y se podrá enlazar el perfil de wikipedia

\begin{figure}[]
    \centering
    \includegraphics[width=1.0\textwidth]{prototipos/Watcher Profile.png}
    \caption{Prototipo de página de perfil de watcher}
    \label{PrototipoWatchersProfile}
\end{figure}

\subsection{Página principal \textbackslash home}
Para usuarios autenticados, les permite ver diferentes secciones relevantes para ellos

\begin{figure}[]
    \centering
    \includegraphics[width=1.0\textwidth]{prototipos/Index Auth.png}
    \caption{Prototipo de vista principal de un usuario autenticado}
    \label{PrototipoHomePage}
\end{figure}


Secciones del home
\begin{enumerate}
    \item Watched \\ Muestra una lista y un pequeño abstracto de los artículos que el usuario hace watching y tienen una visualización actualizada
    \item Queue \\ Muestra una lista de los artículos que le usuario solicito para hacer una visualización pero que el server no ha podido solicitar
    \item Controversial \\ Muestra una lista de aquellas visualizaciones que provocan discusión en la comunidad, caracterizado por muchos comentarios
\end{enumerate}

\subsection{Landing page \textbackslash}
TODO: poner imagen
Esta pagina se encarga de explicar las motivaciones de la App y dar una noción básica del funcionamiento para los watchers.
Tiene el objetivo de cap

\subsection{Página de configuración \textbackslash settings}
TODO: Terminar seccion de settings
Acá el usuario puede bla bla

\subsection{Página de artículo \textbackslash article}
\url{/article?title=<title>&url=<url>}
Muestra un articulo junto con su metadata y las visualizaciones creadas por los usuarios.d

\begin{figure}[]
    \centering
    \includegraphics[width=1.0\textwidth]{prototipos/Article Page.png}
    \caption{Prototipo de página de artículos}
    \label{PrototipoSettingsPage}
\end{figure}

\subsection{Sobre nosotros \textbackslash about}
Breve resumen del proyecto, incluye el documento de tesis y documento de seminario asi como datos de contacto y repositorios de github para futuros contribuyentes.

\section{Modales}

Query Params Globales

?auth

Si este parámetro esta en el url. Presenta la modal de autenticación

?

\section{Planificación de las actividades}

TODO: UNA TABLA DE DOS COLUMNAS

ACTIVIDAD, TIEMPO ESTIMADO.

\begin{lstlisting}
    Preparar el entorno de desarrollo 1/2 semana
    Estudiar API Wikimetrics 2.0 (Back-end) 1/2 semana
    Estudiar API MediaWiki 1/2 semana
    Integracion del API Wikimetrics 2.0 (Back-end) 1/2 semana
    Integracion del API MediaWiki 1/2 semana
\end{lstlisting}

\printbibliography

\end{document}