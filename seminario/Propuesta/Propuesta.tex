
\section{Contexto}

% Resumen del estado actual de wikipedia. Situacion alrededor la tesis tiene sentido
% Me va a situar en el problema que sirve de insumo para justicar el TEG

% DEBE indentificar el problema, resuelto por el objetivo general

Con aproximadamente seis millones cuatrocientos mil articulos ( 6.400.000 ) la Wikipedia lidera como la enciclopedia mas extensa del mundo. No solo eso, sino que muy comunmente al googlear sobre algun tema de interes siempre es el primer resultado.
Esto es gracias al esfuerzo colaborativo de miles de 




% Se hablan de soluciones previas y en eso introducimos wikimetrics y el front

Y son estos mismos grupos que han desarrollado soluciones para extender y analizar las wikis.


\subsection*{Herramientas de extension }
\begin{itemize}
    \item API Wikimetrics
    \item
\end{itemize}
% External tools

% presentamos nuestra solucion

Nuestra labor entonces es facilitar este trabajo para ellos dejandoles crear sus propias graficas sobre los articulos que deseen

\section{Objetivo general}
% TODO: refinar
Crear una nueva version del front-end de wikimetrics


\section{Objetivos específicos}
% TUTORIAL: Objetivos mas peque;os que conforman el todo
% Tambien incluye peque;os estudios, decisiones y aprendizajes 

\begin{itemize}{}{}

    \item Implementar una aplicación web responsive que ofrezca las funcionalidades requeridas por un watcher de un wiki y que pueda ser reconocida por los motores de busqueda.

    \item Consumir y extender la API de wikimetrics para desarrollar una aplicacion web que habilite a sus usuarios construir y visualizar graficas

    \item Definir los requerimientos de la aplicacion
    \item Utilizar un metodo ´agil para el desarrollo de la aplicacion.
    \item Realizar el despliegue y puesta en produccion de la aplicacion

\end{itemize}


\section{Justificación}

Wikipedia contiene en si una masiva cantidad de datos "base" - como articulos, eventos, noticias, media y demas. Pero tambien ella misma genera nuevos datos con cada nueva adicion y edicion a su haber documental. Todos estos "rastros" que dejan miles de colaboradores dia a dia son conocidos como metadata.

Esta desde luego tiene un inmenso valor por si misma, y es refinada principalmente por analistas de datos y aficionados que quieren buscar patrones, relaciones o informacion que no es facil o posible de distinguir con ojos humanos.

% TODO: Porque este trabajo es un trabajo util
% Aprovechar la data que genera wikipedia
% wikipedia tiene apis
% sacarle provecho
% apoyar a los watchers en su labor de vigilancia de sus articulos




\section{Distribución del documento}

% MID_TODO: Distribución del documento
