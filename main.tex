\documentclass[12pt,a4paper]{article}
\usepackage[spanish=nohyphenation]{hyphsubst}
\usepackage[spanish]{babel}
\usepackage{selinput}
\usepackage{tikz}
\usepackage{blindtext}
\usepackage{csquotes}
\usepackage{graphicx}
\usepackage[font=footnotesize,labelfont=bf,justification=centering,textfont=it]{caption}
\usepackage{float}
\usepackage{hyperref}
\usepackage{tabularx}
\usepackage{longtable}
\usepackage{listings}
\usepackage{color}
\usepackage{amsmath}
\usepackage{mathtools}
\usepackage{datetime}
\usepackage{listingsutf8}
\usepackage{mwe}
\usepackage{biblatex}

\addbibresource{teg/references.bib}

\makeatletter

\setlength\bibitemsep{0.5\baselineskip}

\definecolor{lightgray}{rgb}{.9,.9,.9}
\definecolor{darkgray}{rgb}{.4,.4,.4}
\definecolor{purple}{rgb}{0.65, 0.12, 0.82}

\newdateformat{mydate}{\shortmonthname[\THEMONTH], \THEYEAR}

\renewcommand*{\thefootnote}{\arabic{footnote}}
\graphicspath{ {images/} }
\title{
	{Desarrollo de un editor de visualizaciones de propiedades de historiales de wikis}\\
	{\large Universidad Central de Venezuela}\\
}
\author{Adrian J. Mejias. O.}



\addto\captionsspanish{
  \renewcommand{\listfigurename}
    {índice de figuras}
  \renewcommand{\listtablename}
    {índice de tablas}
  \renewcommand\lstlistingname{Código Fuente}
  \renewcommand\lstlistlistingname{índice de código fuente}
}
\renewcommand\spanishtablename{Tabla}
\newcommand{\tabitem}{~~\llap{\textbullet}~~}

\begin{document}

  \begin{titlepage}
	\centering
	{\scshape\LARGE Universidad Central de Venezuela\par}
	{\scshape\LARGE Facultad de Ciencias\par}
	{\scshape\LARGE Escuela de Computación\par}

	\vspace*{\fill}
	{\huge\bfseries Desarrollo de una aplicación distribuida para la extracción, almacenamiento y procesamiento del historial de artículos wiki basados en MediaWiki\par}
	\vspace{2cm}
	{\large TRABAJO ESPECIAL DE GRADO PARA OPTAR AL TÍTULO DE LICENCIADO EN COMPUTACIÓN\par}
	\vspace*{\fill}

	{\large Adrian J. Mejias O. C.I: 25.847.731\par}
	{\large Tutor: Prof. Eugenio Scalise\par}
	\vspace{1cm}
\end{titlepage}

  \section*{Resumen}

% Tıtulo:
% Desarrollo de un editor de visualizaciones de propiedades de historiales de
% wikis.
% Autor:
% Leonardo Testa.
% Tutor:
% Prof. Eugenio Scalise.
% Un Wiki es un sitio web, generalmente de car ́acter informativo (como lo es
% Wikipedia), que puede ser modificado por m ́ultiples personas. Cada una de
% estas modificaciones son almacenadas, y en conjunto conforman un historial
% de versiones, en donde cada versi ́on representa una modificaci ́on y los efectos
% que caus ́o en el art ́ıculo wiki. Siendo Wikipedia un caso real con bastante

% popularidad, es normal que el historial de versiones de un art ́ıculo sea su-
% ficientemente extenso y complejo, por lo tanto las personas interesadas en

% mantener el art ́ıculo “sano” perder ́an una gran suma de tiempo revisando
% las modificaciones. En este documento, presentaremos la investigaci ́on y la
% realizaci ́on de una herramienta web que facilita la lectura de propiedades del
% historial a aquellas personas interesadas, en donde se optar ́a por visualizaci ́on

% de datos como estrategia, de esta forma, mediante una interfaz capaz de ma-
% nipular gr ́aficas el usuario podr ́a proyectar distintas propiedades y conseguir

% f ́acilmente informaci ́on m ́as completa y concretar patrones.
% Palabras claves:
% Visualizaci ́on de datos, wiki, propiedades de historiales, gr ́aficas, herramienta
% web, editor de visualizaciones, wikipedia.





  \tableofcontents

  \listoffigures

  \section{Introduccion}

\subsection{Objetivo general}

\subsection{Objetivos específicos}

\subsection{Justificación}

\subsection{Distribución del documento}



  
paja paja paja

\cite{OperaSVGCanvas}

  \section{Marco Técnologico}




\subsection{Librerias o frameworks para aplicaciones intensivas de frontend }

\section{Arquitectura}

Para que una aplicacion sea descubierta y usada por internautas es fundamental que tenga una buena relacion con los motores de busqueda. 

Sin embargo tambien para asegurar la larga vida y mantenibilidad de la aplicacion y la facilidad de desarrollo se debe tomar en cuenta herramientas extensamente empleadas contemporaneamente como Angular, React y Vue. 

El problema entonces recae en que estas teconologias son meramente para SPA. Lo que implica entonces que no existe una nocion "real" de seo - En las SPA el enrutamiento ocurre del lado del cliente usando javascript, y en consecuencia los crawlers de los motores de busqueda no saben interpretar estas paginas.

Como remedio surge un nuevo paradigma, que es el que vamos a usar para esta aplicacion, conocido como Server Side Rendering; donde se utiliza estas tecnologias SPA como un motor de plantillas para retornar un HTML que los motores de busqueda puedan entender, y despues por un proceso conocido como hydration, las aplicaciones en el lado del cliente dejan de comportarse como HTML plano y retoman sus funcionalides de SPA.

Asi entonces llegamos al perfecto balance en el que tenemos herramientas actuales y faciles de usar, que tambien cumplen con los requerimientos de los motores de busqueda para indexar nuestras paginas. 

  \section{Marco Aplicativo}

Revision de metodologias agiles vs pesadas, usos comunes y puntos a favor para seleccionar agile

agiles es mejor, comparemos entre ellas
\cite{ComparacionFrameworks}
\subsection{Revisión de metodologías ágiles para web }
% POST_SEMINARIO: Comparar y definir

\subsubsection{Lean}
\subsubsection{Scrum}
\subsubsection{TDD}
\subsubsection{RAD}

Requerimientos
watchpage/:watcherId: Home page debe ser personalizable por los watchers. (tambien debe ser un perfil publico)


Posibles adiciones

\begin{list}{}{}
    \item Dejar de manejar los usuarios internamente y manejar la autenticaci´on
          con el API de Wikipedia. Lo principal ser´ıa cumplir con los requerimientos que pide MediaWiki como API para crear la aplicaci´on usando
          OAuth2. De esta forma solo manejaremos un inicio de sesi´on usando el
          usuario de Wikipedia.
    \item Habilitar una opci´on para poder sincronizar los art´ıculos del watchlist
          del usuario de Wikipedia con los art´ıculos de esta plataforma.
          Analizar qu´e visualizaciones pueden ser frecuentes en los usuarios y
          ofrecerlas como visualizaciones predeterminadas.
    \item Implementar la gr´afica original de History Flow. Esta visualizaci´on requiere de mucho c´omputo, por lo que calcularlas en el front-end es
          inviable. La idea es delegarle esta funcionalidad a un servicio en el
          back-end que env´ıe los datos preparados para la visualizaci´on.
    \item Alguna funcionalidad para compartir dashboard de visualizaciones o
          visualizaciones individuales en modo de solo lectura
          79
    \item Implementar un servicio en el backend que se encargue de actualizar
          el estado de los art´ıculos extra´ıdos en el API de Usuarios. De esta
          manera, se evitan constantes peticiones HTTP en la aplicaci´on web
          (HTTP Polling).
\end{list}


  \include{teg/Conclusiones.tex}

  \printbibliography

\end{document}